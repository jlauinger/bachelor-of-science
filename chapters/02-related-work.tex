%% ---------------------------------------------------------------------------------------------------------------------

\chapter{Related Work}\label{ch:relatedwork}

\glsresetall % Resets all acronyms to not used


This is currently a list of notes on papers and books I read, both for summarizing and working on this section in the future.


%% ---------------------------------------------------------------------------------------------------------------------

\section{ieee2012}

That is the 802.11 MAC and PHY standard specification \cite{ieee2012}. Useful for citations and stuff.


%% ---------------------------------------------------------------------------------------------------------------------

\section{perahia2013}

Next Generation Wireless LANs \cite{perahia2013}: Comprehensive overview on the IEEE 802.11 standards, such as a, b, g, n, and ac PHYs and MAC.\\

\textbf{Interesting chapters:}

\begin{itemize}
	\item Chapter 1.1 Overview
	\item Chapter 4.1 Packet structure
	\item Chapter 8.4 Data/ACK frame exchange
	\item Chapter 8.5 Hidden node problem
	\item Chapter 12.1 Frame format
\end{itemize}

802.11 inherits 802 reference structure, such as 6 byte device addresses. 802.11 PHY is the physical layer, 802.11 medium access control (MAC) and 802.2 logical link control (LLC) form the data link layer. MAC uses carrier sense multiple access (CSMA): before transmitting a station listens to the medium. Ethernet uses CSMA/CD (collision detection), whereas 802.11 uses CSMA/CA (collision avoidance). Collision detection through lack of acknowledgements.\\

\textbf{PHYs (overview in \cite{perahia2013} table 1.2):}

\begin{itemize}
	\item IR
	\item 2.4 GHz frequency hopped spread spectrum (FHSS)
	\item 2.4 GHz direct sequence spread spectrum (DSSS)
	\item 802.11b: 2.4 GHz DSSS with complementary code keying (CCK), 11 Mbps
	\item 802.11a: 5 GHz OFDM, 54 Mbps
	\item 802.11g: 2.4 GHz OFDM, 54 Mbps
	\item 802.11n: 5 GHz, 300 Mbps (20 MHz channels), 600 Mbps (40 MHz channels)
	\item 802.11ac: 5 GHz, 1733 Mbps (80 MHz channels, 4 streams), 6933 Mbps (160 MHz channels, 8 streams)
\end{itemize}

Stations within a certain coverage area form a basic service set (BSS). Independent (ad-hoc) or infrastructure (AP). Infrastructure BSSs can be inter-connected as a distribution system (DS). Such a system is called extended service set (ESS). Often Ethernet is used for DS, so that wireless stations can directly address wired PCs in the LAN.

A spatial stream is an independent data stream transmitted or received by an antenna. A device need to have at least as many antennas as spatial streams.\\

\textbf{802.11a PHY packet structure:} see \cite{perahia2013} figure 4.1. STF-LTF-SIG-SRV-Data-Padding. The Short Training Field (STF) is used for start-of-packet detection and automatic gain control (AGC). The Long Training Field (LGF) is used for channel estimation, more precise frequency offset estimation, and time synchronization. The Signal Field (SIG) contains the rate and length information of the packet (BPSK / 64-QAM, $\frac{1}{2}$ / $\frac{3}{4}$ encoding). The Data Field consists of 16 bits Service Field (SRV) and the payload.

STF: 8 $\mu$s, 10 repetitions of a 0.8 $\mu$s symbol. Frequency Domain symbol using 12 of 64 subcarriers. 20 Msample/s yield 2.3 $\mu$s signal which repeats 4 times. This is repeated 2.5 times. Chosen to have good correlation properties and a low peak-to-average power, meaning its properties are preserved after clipping.

LTF: 8 $\mu$s, 2 3.2 $\mu$s training symbols and 1.6 $\mu$s cyclic prefix, 52 subcarriers populated with either 1 or -1, DC unused. Longer symbols mean more precise frequency offset estimation by correlating the two training symbols. Channel estimation after FFT using the basic communication systems equation.

SIG: 24 information bits (4 rate, 1 reserved, 12 length, 1 parity, 6 tail). BPSK and rate $\frac{1}{2}$ binary convolution code (BCC). Nearby stations also decode the SIG to infer correct deferring of transmissions. 4 $\mu$s, 1 64 standard OFDM symbol and cyclic prefix, 52 subcarriers used, 4 pilot. Tail bits are set to 0, used to flush the encoder and decoder.

Data: 16 bit service field, data bits, 6 tail bits (zero), padding. Stream of standard 4 $\mu$s symbols, 52 subcarrier, 4 of which are pilots.

Service: 7 bit scrambler initialization sequence. Used to synchronize descrambler, initially 0. After that: 9 reserved bits set to 0.

Scrambler: in 802.11a it is a length-127 frame synchronous scrambler derived by the polynomial $G(D)=D^7+D^4+1$. Sequence repeatedly generated using all-1-initial: 00001110 11110010 11001001 00000010 00100110 00101110 10110110 00001100 11010100 11100111 10110100 00101010 11111010 01010001 10111000 1111111. When transmitting, the initial scrambler state is set to a pseudo-random non-zero state for each packet.´

The data field is encoded with the following process: bit stream => prepend service bits, append tail and pad bits => scramble => convolutional encoding => grouping into symbols => interleaving => modulator => insert pilots => IFFT => prepend CP => pulse shaping filter => DAC => upmixing => amplification => antenna.

Receiving of the data field: remove guard interval (GI) => FFT => channel equalization => phase rotation correction using pilot carriers => demodulation => deinterleaving => descrambling using first 7 bits of service field as initial state.

In high throughput operations, the preamble consists of a legacy preamble which is compatible to 802.11a (STF, LTF, SIG). After that there is a HT preamble which consists of HT-SIG1, HT-SIG2, HT-STF, HT-STF-1, ... HT-STF-N.\\

\textbf{Medium access control:}

Service Data Unit (SDU): chunk of data handled on a particular OSI layer in service to the layer directly above, e.g. MSDU or PSDU. Beacon frames transmitted by AP in regular time intervals to allow clients to discover a service set.

Distributed coordination function (DCF) (see also \cite{bianchi2000}): sensing the medium idle for a DIFS (DCF inter-frame space). While a station has access to the medium, it separates frames by a SIFS (short inter-frame space), but other stations will not capture the medium because they must wait for a DIFS which is longer. Virtual carrier sense mechanism is called NAV (network allocation vector). Frames may be prepended by a RTS/CTS handshake sequence to minimize effects of collisions. The SIFS is comprised of the PHY Rx latency, MAC processing, and PHY Tx latency. It's 16 $\mu$s. Slot time is 9 $\mu$s. The DIFS is a SIFS and 2 slots long.

After the medium becoming idle, a station must wait a random backoff time before transmitting. This is to reduce collisions caused by multiple waiting stations transmitting. The backoff time is pseudo-randomly chosen out of the [0..CW] interval where CW is the so-called contention window (CW). On each unsuccessful transmit, CW is doubled, yielding exponential growth. After a successful transmission CW is reset.

802.11 MAC employs positive frame acknowledgements by layer 2 ACKs. Without an ACK the sender retransmits the frame. Large MSDUs are fragmented into multiple PSDUs (individually acked) to reduce collision impact. To detect duplicate frames, the data frame includes a retry bit a sequence number field.

Hidden node problem: occurs when a station sees the AP but not a third station that is currently transmitting to the AP. The medium is busy, but the first station senses idle and begins a transmission. Therefore, the transmission to the AP collides. This is mitigated using RTS/CTS and the NAV, since the third station will receive the AP's CTS frame (although it can't hear the RTS) and updates its NAV to reflect the medium being busy for the length of the frame.\\

\textbf{MAC frame format:}

A MAC frame has a MAC header, a variable-length body, and a frame check sequence (FCS) containing a CRC. The MAC header consists of 2 byte Frame Control, 2 byte Duration/ID, 18 bytes Address 1, 2, and 3, 2 bytes Sequence Control, 6 bytes Address 4, 2 bytes QoS Control, and 4 bytes HT control. Most of the time the address fields contain the receive address (RA), transmit address (TA), and the BSSID. The Frame Control part consists of 2 bits protocol version, 2 bits type and 4 bits subtype, and various flags. See \cite{perahia2013} figure 12.2.

The type and subtype fields indicate the frame type, e.g. RTS, CTS, Data, or Beacon. The Sequence Control field is used for retransmission as described above. CTS and ACK frames do not carry a transmitter address, as they are merely sent in direct response to a previous frame.


%% ---------------------------------------------------------------------------------------------------------------------

\section{bianchi2000}

Performance Analysis of the IEEE 802.11 Distributed Coordination Function \cite{bianchi2000}.\\

\textbf{Distribution Coordination Function:}

Random access scheme for 802.11 MAC, based on CSMA/CA with retransmission scheduling based on binary exponential backoff rules. Positive acknowledgement of transmission using ACK frames. RTC/CTS: send short frame prior to data transmission, blocking the channel. Therefore if a collision occurs, a smaller amount of time is wasted.

Point Coordination Function: alternative mode of operation, the AP controls who can transmit. Not commonly used.

Problem of hidden terminals: some pairs of stations can not hear each other, therefore they can not always sense a busy medium.

A station transmits after the medium is sensed idle for a distributed inter-frame space (DIFS). If it is sensed busy, the station waits until a DIFS idle. Before transmitting, the station then also waits for a random backoff time to allow other stations to kick in. The backoff time is measured in slots, where one slot is the time needed to detect another transmit (Rx Tx turnaround time, MAC processing).

The backoff time is chosen out of the interval (0, w-1) where w is the contention window. Starting at CW\_MIN, w is doubled on each unsuccessful transmission, until CW\_MAX. Upon successful transmission, w is reset. When the medium is detected busy, w is frozen. ACKs are transmitted after a short inter-frame space (SIFS).\\

\textbf{Saturation Throughput / Performace:}

Maximum load that the system can carry in stable conditions. Can be used for a long time, unlike the Maximum Throughput which tends to lead to rapid performance decrease.

\cite{bianchi2000} develops a Markov model to derive throughput estimations and verifies them against simulations. The paper contains a lot of figures and tables with the results, but these are not really relevant for me.


%% ---------------------------------------------------------------------------------------------------------------------

\section{gollakota2008}

ZigZag Decoding: Combating Hidden Terminals in Wireless Networks \cite{gollakota2008}.\\

When CSMA fails, such as with hidden terminals, the 802.11 MAC fails: sender repeatedly collide, or one captures the medium. RTS/CTS is not used as a counter measure due to its negative effects on throughput. Observations: senders tend to collide again on the exact same packets, and due to jittering collisions start with a random stretch of interval. With ZigZag, a station finds a part of the packet that is collision-free in one instance, and subtracts that from the collision in the second instance, hence decoding a bit. This bit can then again be subtracted from the first collision, and so on.

With ZigZag, senders do not need to make a trade-off between resilience to collisions and throughput, instead they transmit at the saturation rate. Without a collisions, demodulating packets is unchanged with ZigZag. In the presence of a collision ZigZag achieves an information rate of R, which in theory is the maximum rate possible. ZigZag is modulation-independent, backwards-compatible, and generalizable to a collision with more than two packets.

Unlike Successive Interference Cancellation (SIC), ZigZag works when the channel uses a data rate close to the maximum possible at the given SNR.

If two senders collide, their signals are added up. Problems with standard-decoding of received symbols: Frequence offset and phase tracking (estimate delta f and compensate), Sampling offset (sender and receiver cannot sample the interpolated continuous signal at the exact same sampling points), Inter-symbol interference (ISI, linear equalizer to mitigate effects of neighboring symbols). ZigTag receiver kicks in after receiving a packet failed (wroon checksum).  It will then look for a second packet to check whether there was a collision.

Detecting a packet start is done by correlating the signal with the known 802.11 preamble. Correlation is done on a shifting starting sample, such that the highest correlation value will indicate the sample at which the preamble starts. Correlation is almost zero except when the preamble is perfectly aligned because the preamble sequence is stochastically independent from itself. Two spikes indicate a collision and also show the delta in starting time. The AP maintains coarse estimates of frequency offset for each client.

The AP stores the latest unmatched collisions to map them to possible future collision after retransmission. Collisions are aligned to the point where the second packet starts, and correlated to the stored data. If the packet is the same, correlation will spike.

AP takes the first interference-free chunk of the packet, decodes it, re-encodes the symbol generating an image of the chunk, and subtracts that from the collision, thus deriving another interference-free chunk of the other packet. This process is repeated iteratively until all chunks are decoded.

Compute the ZigZag system parameters using correlation, and inverse filters. Decoding errors propagate until the wrong symbol is by chance right. With BPSK, this has probability $\frac{1}{3}$, meaning errors die exponentially fast. Higher modulation schemes are more affected by errors. Better resilience by applying ZigZag forwards and backwards, picking the decoded symbol with higher confidence.

AP can decode collision packets fast enough to send ACK frame, inserts random padding transmission to prevent hidden terminal from sending while a third station is still transmitting. ZigZag can also decode collision involving any number of senders.


%% ---------------------------------------------------------------------------------------------------------------------

\section{basha2005}

Hardware Implementation of an 802.11a Transmitter \cite{basha2005}.\\

The authors implement a 802.11a transmitter as ASIC. The report gives comprehensive description on the implementation of crucial parts of the packet modulation process.

Scrambler: 7 shift registers, initialized with the scrambler seed. The first register is fed with a feedback signal, generated by XORing the 4th and 7th register. The feedback signal is XORed against the input data as bit stream. The trailing padding bits are scrambled, too, but afterwards reset to zero. This automatically resets the convolutional encoder for the next packet.

Convolutional Encoder: 7 shift registers, 3 XORs. The input bits are inserted into the pipeline and pass through. First XOR input bit, 3rd, 4th, and 7th register. This value is XORed against the 2nd register to produce the ODD bit, and against the 6th register to produce the EVEN bit.

Interleaver: block size $N_{cbps}$ of 48, 06, or 192 bits. First step: reorder bits to map to non-adjacent sub-carriers. Second step: map adjacent bits alternately into less and more significant bits of the sub-carrier constellation. Formula (k is bit index, i is index after first step, j is index after second step):

\begin{align}
  i = \frac{N_{cbps}}{16} \cdot (k ~mod ~16) + floor(\frac{k}{16}), ~~k = 0 .. N_{cbps} - 1 \\
  j = s \cdot floor(\frac{i}{s}) + (i + N_{cbps} - floor(16 \cdot \frac{i}{N_{cbps}})) ~mod ~s, ~~i = 0 .. N_{cbps} - 1  \label{eqn:interleaving}
\end{align}

Cyclic prefix: prepend 16 samples, append 1 sample, yielding 81 instead of 64 samples total.


%% ---------------------------------------------------------------------------------------------------------------------

\section{anitha2005}

A perfect scrambler and descrambler for IEEE 802.11g wireless networks \cite{anitha2005}.\\

The paper develops a scrambling algorithm based on multiple packets that provides physical layer security. The scrambling provides means of transmitting a LFSR seed to generate network keys, where clients can only receive the seed near the AP. Minimal receiving errors generate very large bit errors in the seed. Irrelevant for bachelor thesis.


%% ---------------------------------------------------------------------------------------------------------------------

\section{nychis2009}

Enabling MAC protocol implementations on software-defined radios \cite{nychis2009}.\\

MAC layer implementations require precise timing (e.g. for carrier sense), therefore it is difficult to implement on an SDR. \cite{nychis2009} introduces a split-layer architecture to combine timing precision with flexibility of the host CPU. This can be used to incorporate new techniques that operate somewhat between layer 1 and 2, such as ZigZag \cite{gollakota2008}.


%% ---------------------------------------------------------------------------------------------------------------------

\section{lin2009}

Lock step: an algorithm to reduce Wi-Fi Jitter \cite{lin2009}.\\

For continuous and fairly regular data streams from multiple clients, such as in an environment with several VoIP transmissions, the 802.11 MAC exponential backoff can lead to some clients experiencing lower throughput. Lock step is an algorithm introducing pro-active backoff times after successful transmissions, allowing other clients to kick in. With lock step, clients extend carrier sense to measuring the fraction of free slots within 8 continuous slots. After some collisions, all clients will eventually be transmitting one eighth of the time, thus giving a fairer share.


%% ---------------------------------------------------------------------------------------------------------------------

\section{gudipati2011}

Strider: automatic rate adaptation and collision handling \cite{gudipati2011}.\\

Stripping decoder (Strider) is a rate-less code used for e.g. AutoMAC. Strider strives to be collision-resilient, i.e. senders do not need to care for collision as receivers will be able to decode packets even after suffering a collision. For this, a minimum distance transformer (MDT) is used. Modulated symbols are mapped to a larger symbol space, such as the minimum distance between symbols is larger than the required one to be able to decode signals. Hence, Strider automatically achieves the best code rate for a given channel.

Difference to this thesis: Strider would probably allow to decode colliding MAC addresses, but only if the senders are a-priori using the code. This thesis should focus on standard 802.11.

Exact Strider algorithm not read.


%% ---------------------------------------------------------------------------------------------------------------------

\section{gudipati2012}

AutoMAC: rateless wireless concurrent medium access \cite{gudipati2012}.\\

Similar to ZigZag, AutoMAC provides a way to decode collisions. The paper embraces collisions, as the lack of needing to avoid them allows for simpler MAC implementations. The algorithm uses rateless codes (such as Strider) and can therefore deal with packets with different code rates. Changed 802.11 MAC.

Paper not entirely read.


%% ---------------------------------------------------------------------------------------------------------------------

\section{cao2009}

Behavioral learning of exposed terminals in IEEE 802.11 wireless networks \cite{cao2009}.\\

Description of different scenarios regarding hidden terminals. Algorithm to improve client behavior in that case to maximize throughput and minimize collisions. Irrelevant for bachelor thesis.


%% ---------------------------------------------------------------------------------------------------------------------

\section{wang2010}

Efficient wireless broadcasting using onion decoding \cite{wang2010}.\\

Decoding collisions with broadcast transmission would provide much better throughput. However, this is complicated and past work does not apply, e.g. SIC (one packet must be known) or ZigZag \cite{gollakota2008} (packets must be retransmitted at least once, this is not the case with non-acked broadcasts). Onion decoding uses an iterative approach much like ZigZag, where collision are detected by correlating the preambles, the onion decoder needs an offset between the colliding packets, and the first (collision-free) chunk is subtracted from the collision to obtain a new collision-free chunk. Onion does not need a second collision of the same packets though, because it is specifically designed for broadcast collisions, where multiple stations are forwarding the broadcast packet. Hence, both packets in the collision are the same.

Cost-effective broadcast scheduling (CBS): iterative algorithm to determine the next sender for a broadcast packet, non-conflicting and cost-efficient.

This technique does not apply to this thesis, it is a notable work though.


%% ---------------------------------------------------------------------------------------------------------------------

\section{wu2010}

ACR: active collision recovery in dense wireless sensor networks \cite{wu2010}.\\

Active collision recovery tries to make collisions occur in LS-Form, i.e. a long packet is colliding with a short packet. This gives some opportunities, e.g. the receiver can still decode many bits in the long packet, and the partial corrupt pattern (PCP) is predictable. ACR then employs forward error correction (FEC) to decode the packet.

On layer 2, data from layer 3 is actively shaped in long and short packets which are passed to the MAC. The MAC decides randomly whether to first send the short or long packet. Short packets are added a CRC checksum to verify their integrity. Long packets get FEC and CRC.

Modified CSMA to increase LS collision likelihood: non-uniform distribution of contention slots, transmission delay for short packets.

Receiver checks CRC to detect a possible collision.

Since ACR requires a non-standard MAC, it is not applicable to this bachelor thesis.


%% ---------------------------------------------------------------------------------------------------------------------

\section{chen2014}

In-band wireless cut-through: is it possible? \cite{chen2014}.\\

Technique to send a packet through some forwarding stations, which receive and transmit at the same time. Irrelevant for this bachelor thesis.


%% ---------------------------------------------------------------------------------------------------------------------

\section{halperin2007}

Interference Cancellation: Better Receivers for a New Wireless MAC \cite{halperin2007}.\\

CSMA/CA is often over-protective. A better approach is looking at receiver conditions and allowing multiple transmissions at the same time for the right sender and receiver pairs. Interference with other packets is currently treated as noise, although it is highly structured data.

Constellation of a received collision signal is a combined mn-constellation consisting of the pairwise sums of the points in the individual constellations. Decoding can be done by mapping a sample to a pair of symbols corresponding to the closest of the mn joint constellation points.


%% ---------------------------------------------------------------------------------------------------------------------

\section{park2009}

Cross Layer Multirate Adaptation Using Physical Capture \cite{park2009}.\\

Detect the event that, due to physical capture effect, a packet with higher power is received while a packet with lower power is currently being received. This is counted as collision. The paper proposes an algorithm to share the amount of received collision using a new, special packet. Stations calculate the sum of collisions at neighboring nodes and adjust the code rate being used for sending packets based on that information.


%% ---------------------------------------------------------------------------------------------------------------------

\section{zhu2016}

Packet-Level Failure Classification by Characterizing Failure Patterns in Wireless Sensor Networks \cite{zhu2016}.\\

Observation: real-world packet loss can be caused by weak link or collisions, and those can change very fast. This is because a variety of different reasons such as humans in the signal path, external interference, and unpredictable transmissions by other stations. Thus, failure classification must be done at packet level to provide useful information.

Both naive assumptions of only collisions, or weak link, as failure cause wastes resources, i.e. airtime or transmission power.

PLFC collects byte-level RSSI (Received Signal Strength indicator) and packet-level LQI (link quality indicator) and assigns RSSI rise / fall tags to incoming packets. If a packet has LQI higher than a certain threshold and both a rise and fall tag, a collision is assumed, otherwise weak link.

The receiver informs the sender with an elaborated retransmission request. PLFC could reduce retransmission count by about 50 \%.


%% ---------------------------------------------------------------------------------------------------------------------

\section{keene2010}

Collision Localization for IEEE 802.11 Wireless LANs \cite{keene2010}.\\

The paper presents an algorithm to locate the part of a packet that is not decodable due to a collision. While having some performance regressions to to ODFM, it works with 802.11a/g. The algorithm uses a likelihood ratio test to determine whether a symbol suffered from an error and then determines the packet area having the most errors using a sliding window approach. In some cases, the packet can be fully decoded by correcting the bit errors using an error resolving code.

Problems with OFDM: when symbols are partially corrupted due to a narrow band interference, the timing correlation is lost and the collision can therefore not be exactly localized.


%% ---------------------------------------------------------------------------------------------------------------------

\section{chua2016}

Classification of Transmission Events Based on Receive Power Pattern with Self-Tuning Thresholds in Wireless Receivers \cite{chua2016}.\\

Tries to decide whether a packet failure was due to a collision or weak link by examining the power level at the receiver. Observation: when a collision occurs, the power level is not simply the sum of the two transmission, but rather includes some form of jitter due to oscillations in automatic gain control and other hardware with feedback loops.

Detect power blocks: large difference in signal strength between adjacent samples. Detect collision when packet checksum is wrong and there are many peaks in the power level difference data.


%% ---------------------------------------------------------------------------------------------------------------------

\section{lacurts2010}

Measurement and analysis of real-world 802.11 mesh networks \cite{lacurts2010}.\\

SNR is a good indicator for the optimal bit rate, but one can not craft a SNR to bit rate look-up-table. Hidden terminals occur with a higher probability if the used bit rate is higher.

Paper not entirely read, not really relevant for bachelor thesis.


%% ---------------------------------------------------------------------------------------------------------------------

\section{leonardi2013}

A new adaptive receiver-initiated scheme for mitigating starvation in wireless networks \cite{leonardi2013}.\\

CSMA/CARD: Collision avoidance with receiver detection. Each receiver can predict the presence of any potential sender in a timely fashion. When a collision occurs, the receiver sends a Request for Request to Send (RRTS) packet to expedite the RTS/CTS handshake and better distribute network capacity. Rather irrelevant for bachelor thesis.


%% ---------------------------------------------------------------------------------------------------------------------

\section{cassidy2014}

Collision correction using a cross-layer design architecture for dedicated short range communications vehicle safety messaging \cite{cassidy2014}.\\

The paper presents an algorithm to decode collisions in 802.11p inter-vehicle communication safety messages. These messages are re-transmitted several times and collide with no difference in starting times due to strict slots. Hence, ZigZag can not be applied. The authors added a cache for recent successfully received packets and subtract those packets from incoming collisions, decoding the new, collided packet. Irrelevant for bachelor thesis.


%% ---------------------------------------------------------------------------------------------------------------------

\section{hejazi2010}

Throughput analysis of multiple access relay channel under collision model \cite{hejazi2010}.\\

Presents different strategies for relaying packets in a scenario where multiple stations are relaying and no collisions are allowed. The paper compares different scheduling algorithms and provides some numerical simulations for their maximum throughput.

Irrelevant for this bachelor thesis, paper not entirely read.


%% ---------------------------------------------------------------------------------------------------------------------

\section{zhang2010}

Chorus: collision resolution for efficient wireless broadcast \cite{zhang2010}.\\

Similar to Onion Decoding \cite{wang2010}. Chorus is an algorithm to decode collisions between multiple instances of the same packet, i.e. broadcasts. The broadcast packets will probably start with a slight offset. Chorus takes the first, collision-free chunk and then iteratively subtracts it from the collision to get another collision-free chunk. It is also comparable to ZigZag \cite{gollakota2008}.


%% ---------------------------------------------------------------------------------------------------------------------

\section{choi2013}

Carrier sensing multiple access with collision resolution (CSMA/CR) protocol for next-generation wireless LAN \cite{choi2013}.\\

RTS/CTS is commonly not useful in modern real-world WLAN networks because hidden or exposed terminals do not typically exist. The network is shielded by walls and has relatively few stations. CSMA/CR is a proposed MAC change.

If a collision is detected, one station is allowed priority access to retransmit, so that there won't be a follow-up-collision. Jamming signals.


%% ---------------------------------------------------------------------------------------------------------------------

\section{jibukumar2015}

Impact of capture effect on receiver initiated collision detection with sequential resolution in WLAN \cite{jibukumar2015}.\\

Somewhat similar to CSMA/CR in that when a station detects a collision, it sends out a jamming signal to inform all transmitters that they need to stop. After a collision, the collided packet gets priority access. "The average number of stations undergoing collision in a transmission slot is 2 - 3".

Transmitting stations will send a header, wait a short time, and then send the packet. If a (third) station detects a collision between multiple header transmitters, it sends a jamming signal that is recognized by the transmitters during the sensing window. In that case, the transmitters will wait a random amount of time that is shorter than a DIFS (to prevent new stations to interfere with the collision resolution algorithm). There should now be one of the transmitters solely sending on the channel.

The protocol is called Receiver initiated Fast Sequential collision Resolution (RFSR).


%% ---------------------------------------------------------------------------------------------------------------------

\section{vermeulen2016}

Performance analysis of in-band full duplex collision and interference detection in dense networks \cite{vermeulen2016}.\\

Full Duplex communication is possible for 802.11 with some adjustments to the hardware. This paper presents a MAC change to effectively reduce collisions using full duplex. A receiving station continuously transmits acknowledgements while it receives a packet. This is a simultaneous transmission. If the node senses a collision, it stops sending the ACK, and the transmitter will notice that and stop transmitting. The ACK thus acts like a continuous CTS packet, letting other stations know of the transmission and solving the hidden node problem.


%% ---------------------------------------------------------------------------------------------------------------------

\section{meng2015}

On robust neighbor discovery in mobile wireless networks \cite{meng2015}.\\

New algorithm for neighbor discovery in 802.11 networks that does not use Beacon frames that are decoded, but instead uses a new frame format that only needs to be correlated. Therefore, it saves energy and is more resilient to situations where decoding the packet is not possible.

The frames start with a fixed, independent training sequence called RCover to identify a neighbor discovery message, similar to packet detection using the Short Training Field. After that, a two-stage identification pattern follows. The first stage is a fixed independent Gold-code signal, shifted by a unique amount of samples, randomly chosen by the sending station. Should two stations choose the same offset, are they distinguished by the second stage, a hash of the MAC address. The second stage is not quite as independent and distinguishable and therefore only acts as tie-beaker for identical first stages.

Receivers are triggered by the RCover and correlate the first identity stage to the list of known identities. After a match is found, the second stage is compared. If both stations mismatch, the station is introduced into the list as new transmitter. If a match occurs, the list is updated to include the version with higher SINR.

The detection of sequences is done by cross-correlation at the receiver. The SINR is estimated by self-correlation. If a match is detected at sampling point k, the following holds:

\begin{align}
	C(s,y,\Delta) = \sum_{i=1}^L (s_i^* \cdot y_{i+\Delta}) \approx \sum_{i=1}^L (h_{i+\Delta} \cdot \|s_i\|^2)
	\label{eqn:correlation}
\end{align}


%% ---------------------------------------------------------------------------------------------------------------------

\section{zhao2015}

Collision tolerance: improving channel utilization with correlatable symbol sequences in wireless networks \cite{zhao2015}.\\

Improve network performance by rescheduling collided packets, initiated by the receiver. Each sender picks a random, self-independent correlatable sequence of samples. If a collision is detected by the receiver, it broadcasts a request for these sequences. It then decodes those in parallel using cross-correlation and broadcasts a scheduling packet to all identified senders at once. The senders are assigned a slot and can then resend their packet in a well-defined queue, not wasting airtime by unused slots, as would occur with the binary backoff algorithm of CSMA/CA.


%% ---------------------------------------------------------------------------------------------------------------------
%% ---------------------------------------------------------------------------------------------------------------------
%% ---------------------------------------------------------------------------------------------------------------------
%% ---------------------------------------------------------------------------------------------------------------------
%% ---------------------------------------------------------------------------------------------------------------------
%% ---------------------------------------------------------------------------------------------------------------------


\section{Notes}

Scrambler: since 7 bits in SRV field are initialized to zero, after scrambling they resemble the first 7 output bits of the scrambler LFSR. This is because XOR 0 is a no-op. The receiver can directly insert these bits in its LFSR and from there on start iterating.
