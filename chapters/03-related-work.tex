%% -----------------------------------------------------------------------------

\chapter{Related Work}\label{ch:relatedwork}
\glsresetall % Resets all acronyms to not used

In this chapter I present related research and relevant previous work. The proposed technique of recognizing senders in a collision by correlating their MAC addresses in the time-domain has not been described in the literature yet. However, there have been experiments that point in this direction. This chapter is organized in sections dealing with the coordination function, hidden terminals, collision handling, and consequences of the capture effect.


%% -----------------------------------------------------------------------------

\section{Distributed Coordination Function}

As described in section \ref{sec:dcf}, the Coordination Function defines the rules of communication in an IEEE 802.11 network. There are measures to reduce the impact of a collision, such as the two-way handshake using \gls{RTS} and \gls{CTS} frames. This procedure allows stations in the network to request transmission for a specific time. Other stations can overhear this \gls{RTS} and update their \gls{NAV} tables accordingly. If a collision occurs, it will do so on the \gls{RTS} frame. Since this is much shorter than a data frame, the amount of unusable air-time is smaller, as less data has to be discarded.

Giuseppe Bianchi contributed a performance analysis of the Distributed Coordination Function \cite{bianchi2000}. The maximum load that the system can carry in stable conditions is called saturation throughput. The author develops a Markov model to derive throughput estimations and verifies them against simulations. A result is that using the two-way handshake is not useful in practice, as throughput did not increase in experiments comparable to a  typical wireless application. Therefore, \gls{RTS} and \gls{CTS} frames are not used in most network configurations nowadays \cite{bianchi2000, gollakota2008}.\\

ReCoder \cite{meng2015} is a new algorithm for neighbor discovery in 802.11 networks that does not use beacon frames. Instead, a new frame format that only needs to be correlated, not decoded, is introduced. The technique saves energy and is more resilient to situations, such as collisions, where decoding the packet is not possible.

The alternative frames start with a fixed, independent training sequence called RCover. This sequence is used to identify a neighbor discovery message, similar to frame detection using the Short Training Field. After that, a two-stage identification pattern is applied. The first stage is a fixed Gold-code signal with low auto-correlation, shifted by a unique amount of samples randomly chosen by the sending station. In the event that two stations choose the same offset, they are distinguished by the second stage, which a hash of the MAC address. This stage is not as distinguishable and therefore only acts as tie-beaker for identical first stages.

Receivers are triggered by the RCover sequence and correlate the first identity stage to the list of known stations. If a match is found, the second stage is checked to eliminate possible duplicate shift-amounts. If no match could be found, the reference table is adjusted so that the new identity is included.\\

While the ReCoder algorithm provides sender detection, it requires a change to the standard IEEE 802.11 MAC layer, adding new frame types. Therefore, the technique must be implemented in all devices in order to benefit from it. This is often hard to do, or may even be impossible. This thesis tries to detect senders in a collision without any changes to the MAC layer and is therefore different.


%% -----------------------------------------------------------------------------

\section{Hidden Terminals}

Hidden Terminals are situations where the \gls{CSMA/CA} employed by the IEEE 802.11 MAC layer can not correctly detect an ongoing transmission. This can happen when a receiver is in range of two stations, but these stations can not reach each other \cite{perahia2013}.

In this case, it is possible that one sender does not recognize a transmission of the other sender. It begins sending and disturbs communications between the other sender and the receiver. This effectively renders both transmissions lost as the frames can not be decoded any more be the receiver.\\

\gls{CSMA/CA} is often over-protective, meaning there is some air-time which is not used for transmissions. This is due to the fact that carrier sensing is done at the sender. A better approach could be looking at current conditions at the receiver side \cite{halperin2007}. If two stations are hidden from each other, multiple transmissions could be conducted at the same time for the right sender and receiver pairs.\\

Packet-Level Failure Classification by Characterizing Failure Patterns in Wireless Sensor Networks \cite{zhu2016}.\\

Observation: real-world packet loss can be caused by weak link or collisions, and those can change very fast. This is because a variety of different reasons such as humans in the signal path, external interference, and unpredictable transmissions by other stations. Thus, failure classification must be done at packet level to provide useful information.

Both naive assumptions of only collisions, or weak link, as failure cause wastes resources, i.e. airtime or transmission power.

PLFC collects byte-level RSSI (Received Signal Strength indicator) and packet-level LQI (link quality indicator) and assigns RSSI rise / fall tags to incoming packets. If a packet has LQI higher than a certain threshold and both a rise and fall tag, a collision is assumed, otherwise weak link.

The receiver informs the sender with an elaborated retransmission request. PLFC could reduce retransmission count by about 50 \%.

Performance analysis of in-band full duplex collision and interference detection in dense networks \cite{vermeulen2016}.\\

Full Duplex communication is possible for 802.11 with some adjustments to the hardware. This paper presents a MAC change to effectively reduce collisions using full duplex. A receiving station continuously transmits acknowledgements while it receives a packet. This is a simultaneous transmission. If the node senses a collision, it stops sending the ACK, and the transmitter will notice that and stop transmitting. The ACK thus acts like a continuous CTS packet, letting other stations know of the transmission and solving the hidden node problem.


%% -----------------------------------------------------------------------------

\section{Collision Detection and Decoding}

ZigZag Decoding: Combating Hidden Terminals in Wireless Networks \cite{gollakota2008}.\\

When CSMA fails, such as with hidden terminals, the 802.11 MAC fails: sender repeatedly collide, or one captures the medium. RTS/CTS is not used as a counter measure due to its negative effects on throughput. Observations: senders tend to collide again on the exact same packets, and due to jittering collisions start with a random stretch of interval. With ZigZag, a station finds a part of the packet that is collision-free in one instance, and subtracts that from the collision in the second instance, hence decoding a bit. This bit can then again be subtracted from the first collision, and so on.

With ZigZag, senders do not need to make a trade-off between resilience to collisions and throughput, instead they transmit at the saturation rate. Without a collisions, demodulating packets is unchanged with ZigZag. In the presence of a collision ZigZag achieves an information rate of R, which in theory is the maximum rate possible. ZigZag is modulation-independent, backwards-compatible, and generalizable to a collision with more than two packets.

Unlike Successive Interference Cancellation (SIC), ZigZag works when the channel uses a data rate close to the maximum possible at the given SNR.

If two senders collide, their signals are added up. Problems with standard-decoding of received symbols: Frequence offset and phase tracking (estimate delta f and compensate), Sampling offset (sender and receiver cannot sample the interpolated continuous signal at the exact same sampling points), Inter-symbol interference (ISI, linear equalizer to mitigate effects of neighboring symbols). ZigTag receiver kicks in after receiving a packet failed (wroon checksum).  It will then look for a second packet to check whether there was a collision.

Detecting a packet start is done by correlating the signal with the known 802.11 preamble. Correlation is done on a shifting starting sample, such that the highest correlation value will indicate the sample at which the preamble starts. Correlation is almost zero except when the preamble is perfectly aligned because the preamble sequence is stochastically independent from itself. Two spikes indicate a collision and also show the delta in starting time. The AP maintains coarse estimates of frequency offset for each client.

The AP stores the latest unmatched collisions to map them to possible future collision after retransmission. Collisions are aligned to the point where the second packet starts, and correlated to the stored data. If the packet is the same, correlation will spike.

AP takes the first interference-free chunk of the packet, decodes it, re-encodes the symbol generating an image of the chunk, and subtracts that from the collision, thus deriving another interference-free chunk of the other packet. This process is repeated iteratively until all chunks are decoded.

Compute the ZigZag system parameters using correlation, and inverse filters. Decoding errors propagate until the wrong symbol is by chance right. With BPSK, this has probability $\frac{1}{3}$, meaning errors die exponentially fast. Higher modulation schemes are more affected by errors. Better resilience by applying ZigZag forwards and backwards, picking the decoded symbol with higher confidence.

AP can decode collision packets fast enough to send ACK frame, inserts random padding transmission to prevent hidden terminal from sending while a third station is still transmitting. ZigZag can also decode collision involving any number of senders.

Lock step: an algorithm to reduce Wi-Fi Jitter \cite{lin2009}.\\

For continuous and fairly regular data streams from multiple clients, such as in an environment with several VoIP transmissions, the 802.11 MAC exponential backoff can lead to some clients experiencing lower throughput. Lock step is an algorithm introducing pro-active backoff times after successful transmissions, allowing other clients to kick in. With lock step, clients extend carrier sense to measuring the fraction of free slots within 8 continuous slots. After some collisions, all clients will eventually be transmitting one eighth of the time, thus giving a fairer share.

Strider: automatic rate adaptation and collision handling \cite{gudipati2011}.\\

Stripping decoder (Strider) is a rate-less code used for e.g. AutoMAC. Strider strives to be collision-resilient, i.e. senders do not need to care for collision as receivers will be able to decode packets even after suffering a collision. For this, a minimum distance transformer (MDT) is used. Modulated symbols are mapped to a larger symbol space, such as the minimum distance between symbols is larger than the required one to be able to decode signals. Hence, Strider automatically achieves the best code rate for a given channel.

Difference to this thesis: Strider would probably allow to decode colliding MAC addresses, but only if the senders are a-priori using the code. This thesis should focus on standard 802.11.

AutoMAC: rateless wireless concurrent medium access \cite{gudipati2012}.\\

Similar to ZigZag, AutoMAC provides a way to decode collisions. The paper embraces collisions, as the lack of needing to avoid them allows for simpler MAC implementations. The algorithm uses rateless codes (such as Strider) and can therefore deal with packets with different code rates. Changed 802.11 MAC.

Efficient wireless broadcasting using onion decoding \cite{wang2010}.\\

Decoding collisions with broadcast transmission would provide much better throughput. However, this is complicated and past work does not apply, e.g. SIC (one packet must be known) or ZigZag \cite{gollakota2008} (packets must be retransmitted at least once, this is not the case with non-acked broadcasts). Onion decoding uses an iterative approach much like ZigZag, where collision are detected by correlating the preambles, the onion decoder needs an offset between the colliding packets, and the first (collision-free) chunk is subtracted from the collision to obtain a new collision-free chunk. Onion does not need a second collision of the same packets though, because it is specifically designed for broadcast collisions, where multiple stations are forwarding the broadcast packet. Hence, both packets in the collision are the same.

Cost-effective broadcast scheduling (CBS): iterative algorithm to determine the next sender for a broadcast packet, non-conflicting and cost-efficient.

This technique does not apply to this thesis, it is a notable work though.

ACR: active collision recovery in dense wireless sensor networks \cite{wu2010}.\\

Active collision recovery tries to make collisions occur in LS-Form, i.e. a long packet is colliding with a short packet. This gives some opportunities, e.g. the receiver can still decode many bits in the long packet, and the partial corrupt pattern (PCP) is predictable. ACR then employs forward error correction (FEC) to decode the packet.

On layer 2, data from layer 3 is actively shaped in long and short packets which are passed to the MAC. The MAC decides randomly whether to first send the short or long packet. Short packets are added a CRC checksum to verify their integrity. Long packets get FEC and CRC.

Modified CSMA to increase LS collision likelihood: non-uniform distribution of contention slots, transmission delay for short packets.

Receiver checks CRC to detect a possible collision.

Since ACR requires a non-standard MAC, it is not applicable to this bachelor thesis.

Collision Localization for IEEE 802.11 Wireless LANs \cite{keene2010}.\\

The paper presents an algorithm to locate the part of a packet that is not decodable due to a collision. While having some performance regressions to to ODFM, it works with 802.11a/g. The algorithm uses a likelihood ratio test to determine whether a symbol suffered from an error and then determines the packet area having the most errors using a sliding window approach. In some cases, the packet can be fully decoded by correcting the bit errors using an error resolving code.

Problems with OFDM: when symbols are partially corrupted due to a narrow band interference, the timing correlation is lost and the collision can therefore not be exactly localized.

Chorus: collision resolution for efficient wireless broadcast \cite{zhang2010}.\\

Similar to Onion Decoding \cite{wang2010}. Chorus is an algorithm to decode collisions between multiple instances of the same packet, i.e. broadcasts. The broadcast packets will probably start with a slight offset. Chorus takes the first, collision-free chunk and then iteratively subtracts it from the collision to get another collision-free chunk. It is also comparable to ZigZag \cite{gollakota2008}.

Carrier sensing multiple access with collision resolution (CSMA/CR) protocol for next-generation wireless LAN \cite{choi2013}.\\

RTS/CTS is commonly not useful in modern real-world WLAN networks because hidden or exposed terminals do not typically exist. The network is shielded by walls and has relatively few stations. CSMA/CR is a proposed MAC change.

If a collision is detected, one station is allowed priority access to retransmit, so that there won't be a follow-up-collision. Jamming signals.

Collision tolerance: improving channel utilization with correlatable symbol sequences in wireless networks \cite{zhao2015}.\\

Improve network performance by rescheduling collided packets, initiated by the receiver. Each sender picks a random, self-independent correlatable sequence of samples. If a collision is detected by the receiver, it broadcasts a request for these sequences. It then decodes those in parallel using cross-correlation and broadcasts a scheduling packet to all identified senders at once. The senders are assigned a slot and can then resend their packet in a well-defined queue, not wasting airtime by unused slots, as would occur with the binary backoff algorithm of CSMA/CA.


%% -----------------------------------------------------------------------------

\section{Capture Effect}

Cross Layer Multirate Adaptation Using Physical Capture \cite{park2009}.\\

Detect the event that, due to physical capture effect, a packet with higher power is received while a packet with lower power is currently being received. This is counted as collision. The paper proposes an algorithm to share the amount of received collision using a new, special packet. Stations calculate the sum of collisions at neighboring nodes and adjust the code rate being used for sending packets based on that information.

Classification of Transmission Events Based on Receive Power Pattern with Self-Tuning Thresholds in Wireless Receivers \cite{chua2016}.\\

Tries to decide whether a packet failure was due to a collision or weak link by examining the power level at the receiver. Observation: when a collision occurs, the power level is not simply the sum of the two transmission, but rather includes some form of jitter due to oscillations in automatic gain control and other hardware with feedback loops.

Detect power blocks: large difference in signal strength between adjacent samples. Detect collision when packet checksum is wrong and there are many peaks in the power level difference data.

Impact of capture effect on receiver initiated collision detection with sequential resolution in WLAN \cite{jibukumar2015}.\\

Somewhat similar to CSMA/CR in that when a station detects a collision, it sends out a jamming signal to inform all transmitters that they need to stop. After a collision, the collided packet gets priority access. "The average number of stations undergoing collision in a transmission slot is 2 - 3".

Transmitting stations will send a header, wait a short time, and then send the packet. If a (third) station detects a collision between multiple header transmitters, it sends a jamming signal that is recognized by the transmitters during the sensing window. In that case, the transmitters will wait a random amount of time that is shorter than a DIFS (to prevent new stations to interfere with the collision resolution algorithm). There should now be one of the transmitters solely sending on the channel.

The protocol is called Receiver initiated Fast Sequential collision Resolution (RFSR).
