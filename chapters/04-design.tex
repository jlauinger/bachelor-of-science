%% -----------------------------------------------------------------------------

\chapter{Design and Implementation}\label{ch:design}
\glsresetall % Resets all acronyms to not used

This chapter proposes a means of detecting MAC addresses through sample correlation. The algorithm is implemented in Matlab as a proof-of-concept.


%% -----------------------------------------------------------------------------

\section{Detector Block Design}

First sketch of detector design:

\begin{itemize}
	\item Maintain ARP table
	\item Modulate all combinations of N MAC addresses and initial scrambler state
	\item Modulation includes:
	\begin{itemize}
		\item Generate MAC header
		\item Scramble
		\item Convolutionally encode
		\item Interleave
		\item IFFT
		\item Cyclic prefix
		\item prepend preamble and SIG field? Probably not, simply correlate starting on a sample further into the packet
	\end{itemize}
	\item Receive collision
	\item Correlate modulated fragments with collision samples
	\item There should be four peaks, indicating sender and receiver of both packets
\end{itemize}

Problems:

\begin{itemize}
	\item symbol modulation -> do 6 bytes of address data provide enough samples to correlate to? With 64-QAM?
	\item convolutional encoding, puncturing, and interleaving deterministic?
\end{itemize}

Notes:\\

Use channel models B, D, and E (most common).


%% -----------------------------------------------------------------------------

\section{Sample-Range containing MAC Addresses}


%% -----------------------------------------------------------------------------

\section{Matlab Implementation}
