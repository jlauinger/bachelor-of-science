%% -----------------------------------------------------------------------------

\chapter{Evaluation}\label{ch:evaluation}
\glsresetall % Resets all acronyms to not used

In this chapter, the proposed technique for detecting sender MAC addresses during collisions is evaluated using several simulations as well as experiments on WARP software-defined radios.

Possible questions and problems include the fact, that with higher \glspl{MCS}, there could be not enough samples to correlate the rather short 6 byte MAC address. Another interesting issue is the accuracy of the algorithm with increasing amounts of noise and fading.


%% -----------------------------------------------------------------------------

\section{Frequency-Domain Correlation}\label{sec:freqd-correlation}

During the implementation of the MAC address detection technique, I tried using cross-correlation in the frequency-domain. This means that instead of calculating the similarity between samples, complex symbols are correlated without mapping them into the time-domain. With this approach, the Fourier transform must be applied to the received signal.

It turns out to be at least very difficult, if not impossible, to detect senders in case of a collision with frequency-domain correlation. The reason for that is phase shift due to path delays. When a signal is received with an offset in the time-domain, the constellation is changed in the frequency-domain. This can be shown using the Fourier transform equation:

$$ F(f) = \int_{-\infty}^{\infty} s(t) \cdot e^{-2 \pi i f t} dt $$

Here, $ i $ is the imaginary unit, $ F(f) $ is the Fourier transform at frequency $ f $, and $ s(t) $ is the time-domain signal depending on $ t $. A time offset can now be expressed as $ t + \Delta $. Standard power laws then show that this offset introduces a constant multiplication factor on $ F(f) $, depending on $ \Delta $.

$$ F(f) = \int_{-\infty}^{\infty} s(t + \Delta) \cdot e^{-2 \pi i f \cdot (t + \Delta)} dt = \int_{-\infty}^{\infty} s(t + \Delta) \cdot e^{-2 \pi i f t} \cdot e^{-2 \pi i f \Delta} dt = $$
$$ = e^{-2 \pi i f \Delta} \cdot \int_{-\infty}^{\infty} s(t + \Delta) \cdot e^{-2 \pi i f t} dt $$

Since both $ s(t) $ and $ F(f) $ are complex functions, this multiplication is a rotation, or phase shift, in the complex constellation plane. Furthermore, the rotation is different for different frequencies $ f $.

This leads to a received constellation similar to that shown in figure \ref{fig:freqd-corr}. To create this figure, I applied a Rayleigh channel \cite{NEEDED} to a simulated collision signal, introducing a delay of a few nanoseconds. The signals are modulated with BPSK, so ideally the only constellation points should be $ 1 $ and $ -1 $. However, due to the above described rotation, the constellation is basically a circle.

\begin{figure}[ht]
	\centering
	\includegraphics[width=8cm]{gfx/images/freqd-correlation}
	\caption{Cross Correlation of Frequency-Domain Symbols after Channel}
	\label{fig:freqd-corr}
\end{figure}

This kind of path effects is not at all unusual in a IEEE 802.11 transmission. Under normal circumstances, the known preamble is used to calculate the inverted channel matrix of the path effects, as mentioned in section \ref{sec:multipath}.

When receiving a collision, there is however not only one preamble. Instead, the preambles of two different frames, which experienced different path effects, are added together. Since the preambles itself are designed to have a low autocorrelation \cite{NEEDED}, it is not easily possible to separate the two channel effects from each other. Therefore, I can not reverse the phase shift introduced on the OFDM symbols containing the MAC address.

As expected, experiments with cross-correlation in the frequency-domain without channel equalization were unsuccessful. I therefore disregarded this approach.


%% -----------------------------------------------------------------------------

\section{Time-Domain Correlation}

The experiments I did using time-domain cross-correlation are devided in two groups. First, I used Matlab to simulate different field variations and channel effects. Second, I used WARP boards to try the approach in a real-world scenario. The results of these experiments are summarized in the following sections.

When correlating time-domain samples, I always pre-generated reference samples and calculated the correlation only for the subset of samples that contains the MAC address. Section \ref{sec:mac-periods} describes which period of the signal this is.\\

Reference signals are modulated until just after the Fourier transform and adding of cyclic prefix. For the simulations, I used a sampling rate of 20 MHz, which is the default output of the Matlab WLAN System Toolbox. The WARP boards use a sampling frequency of 40 MHz however, so for these experiments I used interpolation to double the rate.

All experiments used realistic real-world MAC addresses, as described in section \ref{sec:real-world-macs}.


%% -----------------------------------------------------------------------------

\section{Real-World MAC Addresses}\label{sec:real-world-macs}

MAC addresses are 6 byte numbers, where the first 3 bytes are a vendor prefix, identifying the company that built the network interface. The remaining 3 bytes are a unique identifier for the specific hardware.

After initial tries with some artificial MAC addresses in the form of AB:CD:EF:12:34:56, I switched to using a sample set of real-world MAC addresses. This allowed for a closer simulation of an actual use-case for sender detection.\\

I collected 64 MAC addresses from devices that were connected to the wireless university eduroam network. The gathering was done in the afternoon on a weekday. I used airodump from the \texttt{aircrack-ng} software suite to dump all MAC addresses on the network into a file. The actual command looked like this:

\begin{lstlisting}[captionpos=b,caption={Capture Real-World MAC Addresses},label=lst:airodump]
airmon-ng start wlp0s20u1
airodump-ng --essid eduroam -a -o csv -w mac-addresses-eduroam.csv wlp0s20u1mon
\end{lstlisting}

It is important to have the wireless network interface set to promiscuitive mode. This mode instructs the driver to hand all received frames to the kernel. Otherwise, only frames that are addressed to or coming from the local station are gathered, while the rest is filtered. This would mean that only the router's and the station's own MAC address would be cached.

In a possible real-time usage scenario of the sender detection algorithm, it would also be important to use promiscuous mode in order to collect all MAC addresses. These are the addresses for which reference signals need to be modulated and cached.


%% -----------------------------------------------------------------------------

\section{Modulation and Coding Schemes}

Intuitively, the overall sender detection accuracy should decrease for higher \glspl{MCS}. Firstly,

\begin{figure}[H]
	\centering
	\setlength\figureheight{5cm}
	\setlength\figurewidth{0.9\textwidth}
	% uncomment the following line to recompile the figure when it changes otherwise a cached version is used
	%\tikzset{external/remake next}
	% This file was created by matlab2tikz.
%
%The latest updates can be retrieved from
%  http://www.mathworks.com/matlabcentral/fileexchange/22022-matlab2tikz-matlab2tikz
%where you can also make suggestions and rate matlab2tikz.
%
\definecolor{mycolor1}{rgb}{0.24220,0.15040,0.66030}%
\definecolor{mycolor2}{rgb}{0.09640,0.75000,0.71204}%
\definecolor{mycolor3}{rgb}{0.97690,0.98390,0.08050}%
%
\begin{tikzpicture}[%
font=\footnotesize
]

\begin{axis}[%
width=0.951\figurewidth,
height=\figureheight,
at={(0\figurewidth,0\figureheight)},
scale only axis,
bar width=0.8,
xmin=0,
xmax=9,
xtick={1, 2, 3, 4, 5, 6, 7, 8},
xlabel style={font=\color{white!15!black}},
xlabel={MCS},
ymin=0,
ymax=1000,
ylabel style={font=\color{white!15!black}},
ylabel={\# experiments},
axis background/.style={fill=white},
title style={font=\bfseries},
title={Correct guesses for varying MCS},
legend style={at={(0.03,0.97)}, anchor=north west, legend cell align=left, align=left, draw=white!15!black},
clip mode=individual,transpose legend,legend columns=2,legend style={at={(0,1)},anchor=north west,draw=black,fill=white,legend cell align=left}
]
\addplot[ybar stacked, fill=mycolor1, draw=black, area legend] table[row sep=crcr] {%
1	0\\
2	0\\
3	0\\
4	0\\
5	8\\
6	0\\
7	75\\
8	296\\
};
\addplot[forget plot, color=white!15!black] table[row sep=crcr] {%
0	0\\
9	0\\
};
\addlegendentry{0 correct}

\addplot[ybar stacked, fill=mycolor2, draw=black, area legend] table[row sep=crcr] {%
1	0\\
2	0\\
3	1\\
4	0\\
5	95\\
6	118\\
7	481\\
8	527\\
};
\addplot[forget plot, color=white!15!black] table[row sep=crcr] {%
0	0\\
9	0\\
};
\addlegendentry{1 correct}

\addplot[ybar stacked, fill=mycolor3, draw=black, area legend] table[row sep=crcr] {%
1	1000\\
2	1000\\
3	999\\
4	1000\\
5	897\\
6	882\\
7	444\\
8	177\\
};
\addplot[forget plot, color=white!15!black] table[row sep=crcr] {%
0	0\\
9	0\\
};
\addlegendentry{2 correct}

\end{axis}
\end{tikzpicture}%
	\caption{Results: Varying MCS for 1000 experiments}
	\label{fig:vary_mcs}
\end{figure}


%% -----------------------------------------------------------------------------

\section{Scrambler initialization}\label{sec:ex-scrambler}

Note: apparently scrambler init is commonly not chosen at random \cite{NEEDED}

\begin{figure}[H]
	\centering
	% uncomment the following line to recompile the figure when it changes otherwise a cached version is used
	%\tikzset{external/remake next}
	\includegraphics[width=0.9\textwidth,height=5cm]{gfx/images/stock-clouds}
	\caption{Results: Varying Scrambler Initialization for 1000 experiments}
	\label{fig:vary_scrambler}
\end{figure}


%% -----------------------------------------------------------------------------

\section{Preceeding Data Variations}

\begin{figure}[H]
	\centering
	% uncomment the following line to recompile the figure when it changes otherwise a cached version is used
	%\tikzset{external/remake next}
	\includegraphics[width=0.9\textwidth,height=5cm]{gfx/images/stock-clouds}
	\caption{Results: Varying Destination MAC Address for 1000 experiments}
	\label{fig:vary_dest}
\end{figure}


%% -----------------------------------------------------------------------------

\section{Channel Models}

Use channel models B, D, and E (most common).

\begin{figure}[p]
	\centering
	\setlength\figureheight{3cm}
	\setlength\figurewidth{0.40\textwidth}
	\begin{tabular}{cc}
		% uncomment the following line to recompile the figure when it changes otherwise a cached version is used
		%\tikzset{external/remake next}
		\subfloat[MCS 0]{\input{gfx/matlab/tikz/vary_tgn-20170609-0142-num_correct-64_addresses-100_experiments-mcs_0.tikz}} &
		\subfloat[MCS 1]{\input{gfx/matlab/tikz/vary_tgn-20170609-0142-num_correct-64_addresses-100_experiments-mcs_1.tikz}} \\
		\subfloat[MCS 2]{\input{gfx/matlab/tikz/vary_tgn-20170609-0143-num_correct-64_addresses-100_experiments-mcs_2.tikz}} &
		\subfloat[MCS 3]{\input{gfx/matlab/tikz/vary_tgn-20170609-0143-num_correct-64_addresses-100_experiments-mcs_3.tikz}} \\
		\subfloat[MCS 4]{\input{gfx/matlab/tikz/vary_tgn-20170609-0143-num_correct-64_addresses-100_experiments-mcs_4.tikz}} &
		\subfloat[MCS 5]{\input{gfx/matlab/tikz/vary_tgn-20170609-0143-num_correct-64_addresses-100_experiments-mcs_5.tikz}} \\
		\subfloat[MCS 6]{% This file was created by matlab2tikz.
%
%The latest updates can be retrieved from
%  http://www.mathworks.com/matlabcentral/fileexchange/22022-matlab2tikz-matlab2tikz
%where you can also make suggestions and rate matlab2tikz.
%
\definecolor{mycolor1}{rgb}{0.24220,0.15040,0.66030}%
\definecolor{mycolor2}{rgb}{0.09640,0.75000,0.71204}%
\definecolor{mycolor3}{rgb}{0.97690,0.98390,0.08050}%
%
\begin{tikzpicture}[%
font=\footnotesize
]

\begin{axis}[%
width=0.951\figurewidth,
height=\figureheight,
at={(0\figurewidth,0\figureheight)},
scale only axis,
bar width=0.8,
xmin=0.5,
xmax=3.5,
xtick={1,2,3},
xticklabels={{Model-B},{Model-D},{Model-E}},
xlabel style={font=\color{white!15!black}},
xlabel={AGWN SNR},
ymin=0,
ymax=100,
ylabel style={font=\color{white!15!black}},
ylabel={\# experiments},
axis background/.style={fill=white},
title style={font=\bfseries},
title={MCS 6},
legend style={legend cell align=left, align=left, draw=white!15!black},
clip mode=individual,transpose legend,legend columns=2,legend style={at={(0,1)},anchor=north west,draw=black,fill=white,legend cell align=left}
]
\addplot[ybar stacked, fill=mycolor1, draw=black, area legend] table[row sep=crcr] {%
1	41\\
2	58\\
3	73\\
};
\addplot[forget plot, color=white!15!black] table[row sep=crcr] {%
0.5	0\\
3.5	0\\
};
\addlegendentry{0 correct}

\addplot[ybar stacked, fill=mycolor2, draw=black, area legend] table[row sep=crcr] {%
1	42\\
2	33\\
3	26\\
};
\addplot[forget plot, color=white!15!black] table[row sep=crcr] {%
0.5	0\\
3.5	0\\
};
\addlegendentry{1 correct}

\addplot[ybar stacked, fill=mycolor3, draw=black, area legend] table[row sep=crcr] {%
1	17\\
2	9\\
3	1\\
};
\addplot[forget plot, color=white!15!black] table[row sep=crcr]  &
		\subfloat[MCS 7]{\input{gfx/matlab/tikz/vary_tgn-20170609-0143-num_correct-64_addresses-100_experiments-mcs_7.tikz}} \\
	\end{tabular}
	\caption{Varying TGn Channel for 100 Experiments}
	\label{fig:vary_tng}
\end{figure}

\begin{figure}[p]
	\centering
	\setlength\figureheight{3cm}
	\setlength\figurewidth{0.40\textwidth}
	\begin{tabular}{cc}
		% uncomment the following line to recompile the figure when it changes otherwise a cached version is used
		%\tikzset{external/remake next}
		\subfloat[MCS 0]{\input{gfx/matlab/tikz/vary_trms-20170609-0248-num_correct-64_addresses-10_experiments-mcs_0.tikz}} &
		\subfloat[MCS 1]{\input{gfx/matlab/tikz/vary_trms-20170609-0248-num_correct-64_addresses-10_experiments-mcs_1.tikz}} \\
		\subfloat[MCS 2]{% This file was created by matlab2tikz.
%
%The latest updates can be retrieved from
%  http://www.mathworks.com/matlabcentral/fileexchange/22022-matlab2tikz-matlab2tikz
%where you can also make suggestions and rate matlab2tikz.
%
\definecolor{mycolor1}{rgb}{0.24220,0.15040,0.66030}%
\definecolor{mycolor2}{rgb}{0.09640,0.75000,0.71204}%
\definecolor{mycolor3}{rgb}{0.97690,0.98390,0.08050}%
%
\begin{tikzpicture}[%
font=\footnotesize
]

\begin{axis}[%
width=0.951\figurewidth,
height=\figureheight,
at={(0\figurewidth,0\figureheight)},
scale only axis,
bar width=0,
xmin=5e-08,
xmax=5.5e-07,
xlabel style={font=\color{white!15!black}},
xlabel={$\text{802.11g stdchan: t}_\text{R}\text{MS in seconds}$},
ymin=0,
ymax=10,
ylabel style={font=\color{white!15!black}},
ylabel={\# experiments},
axis background/.style={fill=white},
title style={font=\bfseries},
title={MCS 2},
legend style={legend cell align=left, align=left, draw=white!15!black},
clip mode=individual,transpose legend,legend columns=2,legend style={at={(0,1)},anchor=north west,draw=black,fill=white,legend cell align=left}
]
\addplot[ybar stacked, fill=mycolor1, draw=black, area legend] table[row sep=crcr] {%
1e-07	0\\
1.5e-07	0\\
2e-07	0\\
2.5e-07	0\\
3e-07	1\\
3.5e-07	4\\
4e-07	1\\
4.5e-07	3\\
5e-07	3\\
};
\addplot[forget plot, color=white!15!black] table[row sep=crcr] {%
5e-08	0\\
5.5e-07	0\\
};
\addlegendentry{0 correct}

\addplot[ybar stacked, fill=mycolor2, draw=black, area legend] table[row sep=crcr] {%
1e-07	7\\
1.5e-07	7\\
2e-07	9\\
2.5e-07	9\\
3e-07	7\\
3.5e-07	5\\
4e-07	7\\
4.5e-07	6\\
5e-07	7\\
};
\addplot[forget plot, color=white!15!black] table[row sep=crcr] {%
5e-08	0\\
5.5e-07	0\\
};
\addlegendentry{1 correct}

\addplot[ybar stacked, fill=mycolor3, draw=black, area legend] table[row sep=crcr] {%
1e-07	3\\
1.5e-07	3\\
2e-07	1\\
2.5e-07	1\\
3e-07	2\\
3.5e-07	1\\
4e-07	2\\
4.5e-07	1\\
5e-07	0\\
};
\addplot[forget plot, color=white!15!black] table[row sep=crcr]  &
		\subfloat[MCS 3]{\input{gfx/matlab/tikz/vary_trms-20170609-0248-num_correct-64_addresses-10_experiments-mcs_3.tikz}} \\
		\subfloat[MCS 4]{\input{gfx/matlab/tikz/vary_trms-20170609-0248-num_correct-64_addresses-10_experiments-mcs_4.tikz}} &
		\subfloat[MCS 5]{\input{gfx/matlab/tikz/vary_trms-20170609-0248-num_correct-64_addresses-10_experiments-mcs_5.tikz}} \\
		\subfloat[MCS 6]{\input{gfx/matlab/tikz/vary_trms-20170609-0248-num_correct-64_addresses-10_experiments-mcs_6.tikz}} &
		\subfloat[MCS 7]{% This file was created by matlab2tikz.
%
%The latest updates can be retrieved from
%  http://www.mathworks.com/matlabcentral/fileexchange/22022-matlab2tikz-matlab2tikz
%where you can also make suggestions and rate matlab2tikz.
%
\definecolor{mycolor1}{rgb}{0.24220,0.15040,0.66030}%
\definecolor{mycolor2}{rgb}{0.09640,0.75000,0.71204}%
\definecolor{mycolor3}{rgb}{0.97690,0.98390,0.08050}%
%
\begin{tikzpicture}[%
font=\footnotesize
]

\begin{axis}[%
width=0.951\figurewidth,
height=\figureheight,
at={(0\figurewidth,0\figureheight)},
scale only axis,
bar width=0,
xmin=5e-08,
xmax=5.5e-07,
xlabel style={font=\color{white!15!black}},
xlabel={$\text{802.11g stdchan: t}_\text{R}\text{MS in seconds}$},
ymin=0,
ymax=10,
ylabel style={font=\color{white!15!black}},
ylabel={\# experiments},
axis background/.style={fill=white},
title style={font=\bfseries},
title={MCS 7},
legend style={legend cell align=left, align=left, draw=white!15!black},
clip mode=individual,transpose legend,legend columns=2,legend style={at={(0,1)},anchor=north west,draw=black,fill=white,legend cell align=left}
]
\addplot[ybar stacked, fill=mycolor1, draw=black, area legend] table[row sep=crcr] {%
1e-07	8\\
1.5e-07	7\\
2e-07	7\\
2.5e-07	8\\
3e-07	9\\
3.5e-07	10\\
4e-07	10\\
4.5e-07	8\\
5e-07	10\\
};
\addplot[forget plot, color=white!15!black] table[row sep=crcr] {%
5e-08	0\\
5.5e-07	0\\
};
\addlegendentry{0 correct}

\addplot[ybar stacked, fill=mycolor2, draw=black, area legend] table[row sep=crcr] {%
1e-07	2\\
1.5e-07	3\\
2e-07	3\\
2.5e-07	2\\
3e-07	1\\
3.5e-07	0\\
4e-07	0\\
4.5e-07	2\\
5e-07	0\\
};
\addplot[forget plot, color=white!15!black] table[row sep=crcr] {%
5e-08	0\\
5.5e-07	0\\
};
\addlegendentry{1 correct}

\addplot[ybar stacked, fill=mycolor3, draw=black, area legend] table[row sep=crcr] {%
1e-07	0\\
1.5e-07	0\\
2e-07	0\\
2.5e-07	0\\
3e-07	0\\
3.5e-07	0\\
4e-07	0\\
4.5e-07	0\\
5e-07	0\\
};
\addplot[forget plot, color=white!15!black] table[row sep=crcr]  \\
	\end{tabular}
	\caption{Varying $t_{RMS}$ in a Standard Channel for 100 Experiments}
	\label{fig:vary_trms}
\end{figure}



%% -----------------------------------------------------------------------------

\section{WARP Experiments}

\begin{figure}[H]
	\centering
	\setlength\figureheight{5cm}
	\setlength\figurewidth{0.9\textwidth}
	% uncomment the following line to recompile the figure when it changes otherwise a cached version is used
	%\tikzset{external/remake next}
	% This file was created by matlab2tikz.
%
%The latest updates can be retrieved from
%  http://www.mathworks.com/matlabcentral/fileexchange/22022-matlab2tikz-matlab2tikz
%where you can also make suggestions and rate matlab2tikz.
%
\begin{tikzpicture}[%
font=\footnotesize
]

\begin{axis}[%
width=0.951\figurewidth,
height=\figureheight,
at={(0\figurewidth,0\figureheight)},
scale only axis,
xmin=0,
xmax=1800,
ymin=0,
ymax=3,
axis background/.style={fill=white},
title style={font=\bfseries},
title={LTF correlation for real-world samples},
axis x line*=bottom,
axis y line*=left,
legend style={legend cell align=left, align=left, draw=white!15!black},
clip mode=individual,transpose legend,legend columns=2,legend style={at={(0,1)},anchor=north west,draw=black,fill=white,legend cell align=left}
]
\addplot [color=blue, line width=1.0pt, mark=*, mark options={solid, blue}]
  table[row sep=crcr]{%
0	0.720931094814423\\
1	1.16568673441402\\
2	1.3141546759545\\
3	0.641645290157508\\
4	0.395108219802532\\
5	1.02254413691229\\
6	1.23346012773391\\
7	1.42940247728357\\
8	1.30855846240446\\
9	0.792541754711272\\
10	0.677926422656566\\
11	0.763008367099554\\
12	0.572326863744454\\
13	0.649448460335741\\
14	0.812836127899684\\
15	0.639897774704796\\
16	0.502144929911757\\
17	0.640000230333206\\
18	0.81890427349694\\
19	1.53125596120855\\
20	1.8355736526179\\
21	1.18894623425786\\
22	0.60757103489733\\
23	0.407157849107528\\
24	0.699168812767027\\
25	1.59107505149703\\
26	1.43988110173639\\
27	1.37478224378645\\
28	2.149245777478\\
29	1.76369670544532\\
30	0.483670341471905\\
31	0.720050971652801\\
32	0.577076089785491\\
33	0.864228058873508\\
34	1.7400151550178\\
35	1.60627196930859\\
36	1.10858033190134\\
37	0.906993644997446\\
38	0.466732145205898\\
39	0.758335526961838\\
40	0.923912129091483\\
41	0.562123267347041\\
42	0.146318849471526\\
43	0.299150577295798\\
44	0.518541495586723\\
45	0.307678574313589\\
46	0.269363873348616\\
47	0.7841954651033\\
48	0.903290445718305\\
49	0.771001575016175\\
50	1.37252584723606\\
51	1.97341433169664\\
52	1.84990733323974\\
53	1.38084133037432\\
54	1.27928436110181\\
55	0.850531224546523\\
56	0.770733316261401\\
57	1.57252407213278\\
58	1.27841331252654\\
59	0.775021234513872\\
60	1.58959359668981\\
61	1.54008487485559\\
62	0.944250193569413\\
63	0.903618738844383\\
64	0.808518293906735\\
65	1.26113739140597\\
66	1.58295659094252\\
67	1.17050995255307\\
68	0.473145617440272\\
69	0.366412679946316\\
70	0.594788181609445\\
71	0.548538264495461\\
72	0.357807390986729\\
73	0.467814304964996\\
74	0.39858933272967\\
75	0.39133352216922\\
76	0.29014892545605\\
77	0.414382894055281\\
78	0.691198625034713\\
79	0.451295808218838\\
80	0.336777310915194\\
81	0.673566047736059\\
82	0.736654379278464\\
83	1.17905398352261\\
84	1.51883016653246\\
85	1.32439126197165\\
86	1.00600801576514\\
87	0.963546970549056\\
88	1.18006580980734\\
89	1.25349188629195\\
90	0.762923878560664\\
91	1.12990732924819\\
92	1.67390574330647\\
93	1.21911492682408\\
94	0.725775917654175\\
95	0.906257359262177\\
96	0.79359366771158\\
97	1.34538942764149\\
98	1.61536202534121\\
99	1.32048539338983\\
100	1.01812234763284\\
101	0.591981636089268\\
102	0.370843752321289\\
103	0.525446350300325\\
104	0.0531063527778062\\
105	0.602782140312682\\
106	0.9925350287863\\
107	0.889246032822645\\
108	0.414779769670687\\
109	0.429045440294952\\
110	0.918347836443727\\
111	0.787004409124193\\
112	0.0640846830193087\\
113	0.70839962170215\\
114	1.03855374671624\\
115	0.772946558469687\\
116	0.475001387401616\\
117	0.696477679643915\\
118	0.812466249307104\\
119	0.868370373915567\\
120	0.77569356256677\\
121	0.520636853334394\\
122	0.817592638653556\\
123	1.20595902508417\\
124	1.07296506081179\\
125	0.698172053895777\\
126	1.01856457142272\\
127	1.10954500520664\\
128	0.791248583502632\\
129	0.948277538125297\\
130	0.962877522329877\\
131	0.511015199610062\\
132	0.316811014975686\\
133	0.243392905032121\\
134	0.247417245739964\\
135	0.60408716366557\\
136	0.515351365618617\\
137	0.140029251031592\\
138	0.281398869611954\\
139	0.479328866858171\\
140	0.492905940122276\\
141	0.467124543000461\\
142	0.430931151562365\\
143	0.254032927394278\\
144	0.393369736537108\\
145	0.610705190118998\\
146	0.599891193561674\\
147	0.534322709031245\\
148	0.506749062094263\\
149	0.606530494360062\\
150	0.786256344223052\\
151	0.842088797648275\\
152	0.717315789990021\\
153	0.396612166978163\\
154	0.590618822960569\\
155	0.661243410011752\\
156	0.494646879338831\\
157	0.528392735902201\\
158	0.664038681511439\\
159	0.635322247949787\\
160	0.746189844141419\\
161	0.796185035737734\\
162	0.444645157646661\\
163	0.105367523231385\\
164	0.462759538553399\\
165	0.528502846102991\\
166	0.447561848084432\\
167	0.272947785051632\\
168	0.161697362821431\\
169	0.21934718394202\\
170	0.161857317123406\\
171	0.126760061054273\\
172	0.104667768489787\\
173	0.245152265595692\\
174	0.383178680584173\\
175	0.401820121385813\\
176	0.384852177380141\\
177	0.364535067975096\\
178	0.297874363959069\\
179	0.405516937768411\\
180	0.585421660014701\\
181	0.617698531786405\\
182	0.546337347184682\\
183	0.533466097046177\\
184	0.623059927555602\\
185	0.603682182501965\\
186	0.3988094949763\\
187	0.200985911686892\\
188	0.36040025909977\\
189	0.588826598028605\\
190	0.682585079801368\\
191	0.679289292813292\\
192	0.761875072505756\\
193	0.698516815309861\\
194	0.39421147625963\\
195	0.276052017302006\\
196	0.328173765579516\\
197	0.352938489363377\\
198	0.49936154530006\\
199	0.478941854678629\\
200	0.316216628578522\\
201	0.284481384532082\\
202	0.305176674189754\\
203	0.24654769175249\\
204	0.113572281764097\\
205	0.202486135424496\\
206	0.28997574217328\\
207	0.208581405848185\\
208	0.108292644688327\\
209	0.169527694743327\\
210	0.297418899441112\\
211	0.429937223568841\\
212	0.43570541297004\\
213	0.4124175737342\\
214	0.471594738112648\\
215	0.405230696692198\\
216	0.406921704076214\\
217	0.575618191362919\\
218	0.48878883605\\
219	0.169130999319071\\
220	0.567054895581535\\
221	0.85262387819276\\
222	0.77816908415513\\
223	0.557048282992595\\
224	0.493846137795923\\
225	0.509416315343237\\
226	0.443475650423118\\
227	0.247407494096242\\
228	0.350565851265824\\
229	0.605479729595615\\
230	0.529929464221523\\
231	0.19690633646066\\
232	0.243364953771876\\
233	0.210329810576645\\
234	0.21937051796833\\
235	0.527037808309991\\
236	0.626557594360931\\
237	0.467382636567737\\
238	0.162633364607986\\
239	0.184995571095381\\
240	0.26246213333078\\
241	0.0821647749139948\\
242	0.270526287960425\\
243	0.437176949343445\\
244	0.302789595703765\\
245	0.171002169950304\\
246	0.316708251796659\\
247	0.276169073916823\\
248	0.332452471436443\\
249	0.469483989661724\\
250	0.385734487612786\\
251	0.200509931930949\\
252	0.25204172112822\\
253	0.504007585976586\\
254	0.701693246140831\\
255	0.824194265991174\\
256	0.886150226418558\\
257	0.839686220654251\\
258	0.65026667423914\\
259	0.389181632780917\\
260	0.395650617968788\\
261	0.616696237425897\\
262	0.718932827668808\\
263	0.647710770975872\\
264	0.550861303398622\\
265	0.450146320767909\\
266	0.412634190553644\\
267	0.579066853708034\\
268	0.597760345697625\\
269	0.426102032379351\\
270	0.337646287985447\\
271	0.328176469575033\\
272	0.259884442540605\\
273	0.106654710453866\\
274	0.208984579514297\\
275	0.348482290942643\\
276	0.299623990902014\\
277	0.185337646753725\\
278	0.248431484437914\\
279	0.319426358645417\\
280	0.2772272175275\\
281	0.266320662016076\\
282	0.302359122889774\\
283	0.1295808165565\\
284	0.256798588657833\\
285	0.618669631925832\\
286	0.747481217358204\\
287	0.668802436681065\\
288	0.521261676168647\\
289	0.387264596851937\\
290	0.421927502752322\\
291	0.412880147967949\\
292	0.44093250357214\\
293	0.740522556061655\\
294	0.84493967584052\\
295	0.70879846409772\\
296	0.527019992968485\\
297	0.446707596894314\\
298	0.922924696516647\\
299	1.33860491392172\\
300	1.16373851032171\\
301	0.550601006176828\\
302	0.457963004040581\\
303	0.583541129603565\\
304	0.37844202335645\\
305	0.148802044337363\\
306	0.0643794430407777\\
307	0.0699834185368895\\
308	0.061950902565579\\
309	0.0019552948287911\\
310	0.0747935165795\\
311	0.192455676578184\\
312	0.241462953288397\\
313	0.192589186111211\\
314	0.239512960969524\\
315	0.105756428246341\\
316	0.278102811389726\\
317	0.597147363484762\\
318	0.60448465931663\\
319	0.464636656167896\\
320	0.43724956929255\\
321	0.403090206001361\\
322	0.457400795038727\\
323	0.482701590105114\\
324	0.570512550744854\\
325	0.670777770100924\\
326	0.526133133681203\\
327	0.438132042371519\\
328	0.558709667575666\\
329	0.526842875064231\\
330	0.547016416661765\\
331	0.686957039296202\\
332	0.619633653152223\\
333	0.446535743665881\\
334	0.401598165478051\\
335	0.26295788991983\\
336	0.140291712994303\\
337	0.205874982385661\\
338	0.169661704066953\\
339	0.0624727817771767\\
340	0.0893189391578162\\
341	0.206600773096695\\
342	0.229952737486357\\
343	0.103883987007097\\
344	0.182407278735709\\
345	0.413230478610313\\
346	0.48156266905976\\
347	0.370049439642338\\
348	0.40538385916249\\
349	0.575515219196783\\
350	0.582529342292222\\
351	0.536169673359967\\
352	0.50223149344944\\
353	0.468877540957649\\
354	0.544389491162411\\
355	0.541534218040889\\
356	0.599045296240793\\
357	0.744769347128761\\
358	0.647465449906921\\
359	0.468725541265581\\
360	0.552110239416768\\
361	0.541555701800614\\
362	0.522590997478641\\
363	0.685577820123613\\
364	0.67601962887479\\
365	0.429600866634372\\
366	0.261877522010782\\
367	0.227408262911671\\
368	0.133497588871437\\
369	0.089945183166874\\
370	0.102587420522226\\
371	0.0804193476926125\\
372	0.0791667341329972\\
373	0.172331402914878\\
374	0.273997943835839\\
375	0.255174555723054\\
376	0.156559063610574\\
377	0.253249409823208\\
378	0.354421169283508\\
379	0.44035592187088\\
380	0.558788816431985\\
381	0.597831387477915\\
382	0.623806910205155\\
383	0.750927920520728\\
384	0.810386708930307\\
385	0.734001927474996\\
386	0.626627199890916\\
387	0.513933611241598\\
388	0.464879776843613\\
389	0.522537413387102\\
390	0.472707109212168\\
391	0.35198434236775\\
392	0.396984895454405\\
393	0.396884367714415\\
394	0.278056664674387\\
395	0.330093375350994\\
396	0.400853525190633\\
397	0.287393979226489\\
398	0.107769990829097\\
399	0.120752691598849\\
400	0.130345260940638\\
401	0.124302926768883\\
402	0.119781298917038\\
403	0.10724548380331\\
404	0.0749766422282917\\
405	0.0696248294056397\\
406	0.0681030770244627\\
407	0.0807744817031463\\
408	0.191314611252934\\
409	0.139263889875701\\
410	0.229041304184165\\
411	0.526275432216956\\
412	0.701293042452557\\
413	0.722321553526931\\
414	0.608695795865329\\
415	0.470250071026404\\
416	0.372382486666876\\
417	0.342238357747492\\
418	0.581727864448338\\
419	0.645478709394455\\
420	0.535502720123082\\
421	0.640355412522363\\
422	0.669522583261143\\
423	0.662384186739298\\
424	0.530603718722175\\
425	0.613766447794406\\
426	1.62962815873112\\
427	2.18742149731996\\
428	1.75063274281811\\
429	0.713207227160355\\
430	0.332529025298972\\
431	0.466168870128803\\
432	0.212629992007919\\
433	0.0696353788421605\\
434	0.0292271723798141\\
435	0.0487973338728497\\
436	0.0378511297748502\\
437	0.145987764444517\\
438	0.163896206279309\\
439	0.135259008653192\\
440	0.107909610136488\\
441	0.0867524219458357\\
442	0.12768496006025\\
443	0.143553785797558\\
444	0.357587974327753\\
445	0.514104735948499\\
446	0.483547763275411\\
447	0.422729230262341\\
448	0.397699201703674\\
449	0.277495362749907\\
450	0.34121589990255\\
451	0.435990378710489\\
452	0.46308636142043\\
453	0.653436524347772\\
454	0.685546696138492\\
455	0.475514977166753\\
456	0.457403743120247\\
457	0.643545047376216\\
458	0.820065664848105\\
459	0.855048760871613\\
460	0.631459686448819\\
461	0.238692606641416\\
462	0.109936427343734\\
463	0.225743573261735\\
464	0.205057250220913\\
465	0.209545612633914\\
466	0.194307456419798\\
467	0.212013010702738\\
468	0.315592792921871\\
469	0.374044718614761\\
470	0.322196908830155\\
471	0.199943557666601\\
472	0.183571035581014\\
473	0.203629935828034\\
474	0.0967061348641033\\
475	0.317954836994668\\
476	0.558435943629889\\
477	0.628958225214659\\
478	0.58599120797271\\
479	0.521177888862574\\
480	0.441692616623721\\
481	0.362237162660365\\
482	0.348123877856604\\
483	0.24667760681855\\
484	0.0927457314340676\\
485	0.390266834901365\\
486	0.506903449647394\\
487	0.329534002725255\\
488	0.231083438639872\\
489	0.321385464652983\\
490	0.301439202980121\\
491	0.423071804787139\\
492	0.510397848551832\\
493	0.441686612223795\\
494	0.306567348541328\\
495	0.229532864078417\\
496	0.212948528430432\\
497	0.344720113604652\\
498	0.751355359009571\\
499	1.19341460357228\\
500	1.32029379390884\\
501	0.982902203486811\\
502	0.457420925332982\\
503	0.448599403332307\\
504	0.509341117371367\\
505	0.268401526425729\\
506	0.0450638880970166\\
507	0.155546304715029\\
508	0.123518368779894\\
509	0.229780171386422\\
510	0.380687471099624\\
511	0.531265332343756\\
512	0.611674281278041\\
513	0.510239718334173\\
514	0.206356913472231\\
515	0.163056622459863\\
516	0.397043288038431\\
517	0.373029910696369\\
518	0.187826502544895\\
519	0.186351304877066\\
520	0.143911842690011\\
521	0.0919027122842002\\
522	0.285106790008328\\
523	0.376514008930879\\
524	0.330948670750887\\
525	0.243736103314185\\
526	0.194453546686479\\
527	0.215527260740158\\
528	0.270914893776744\\
529	0.280952361841279\\
530	0.328851221032298\\
531	0.355120198419307\\
532	0.246621493916162\\
533	0.122421211315998\\
534	0.241784890945979\\
535	0.399879468695546\\
536	0.386981682143234\\
537	0.146802945342198\\
538	0.190101043474684\\
539	0.354635196984993\\
540	0.256362366935009\\
541	0.202145416770583\\
542	0.275975619479487\\
543	0.226249791938257\\
544	0.181515901056431\\
545	0.183981631139009\\
546	0.14700641268006\\
547	0.0767713417808291\\
548	0.353891609790477\\
549	0.396779493948713\\
550	0.065014749739428\\
551	0.471968106924515\\
552	0.4665234658575\\
553	0.699590373039856\\
554	1.64135520528653\\
555	1.91092390697065\\
556	1.20813116917421\\
557	0.125100764508337\\
558	0.492964451692283\\
559	0.345725762430673\\
560	0.119367159623564\\
561	0.243208022465742\\
562	0.106844054970704\\
563	0.441731174060692\\
564	0.515015412294371\\
565	0.304916788848737\\
566	0.123566332112047\\
567	0.11819235961056\\
568	0.0379660573172009\\
569	0.105342995804198\\
570	0.0548198690840899\\
571	0.132003648163404\\
572	0.245283806080487\\
573	0.323736087640489\\
574	0.426908786011443\\
575	0.461868744581769\\
576	0.381357991192025\\
577	0.180615401503067\\
578	0.0422839812668914\\
579	0.0870583734129203\\
580	0.102729756668487\\
581	0.190698642907933\\
582	0.209365093465682\\
583	0.173519112596846\\
584	0.141780768568869\\
585	0.0471237112534095\\
586	0.0839856422532778\\
587	0.0975443658481744\\
588	0.168890661065352\\
589	0.261569354576836\\
590	0.167375007827894\\
591	0.123784536573176\\
592	0.344569095775757\\
593	0.360173966085722\\
594	0.219608993100634\\
595	0.166929239182116\\
596	0.17934005497151\\
597	0.0935966832277601\\
598	0.0822743037972114\\
599	0.174260037661151\\
600	0.204598898824003\\
601	0.10319176931733\\
602	0.199960882689767\\
603	0.460861184832484\\
604	0.538102234209114\\
605	0.385700753282671\\
606	0.168091302189618\\
607	0.170066095147624\\
608	0.150228058831398\\
609	0.173892894895959\\
610	0.32303169823533\\
611	0.294465048718059\\
612	0.150392496197543\\
613	0.357088317376207\\
614	0.534191946235149\\
615	0.465236266689985\\
616	0.229437168196802\\
617	0.1787559957163\\
618	0.0235683872542294\\
619	0.340514543717646\\
620	0.483301905003438\\
621	0.136070481276109\\
622	0.511422722392272\\
623	0.779470773884751\\
624	0.552127012557067\\
625	0.293965169544466\\
626	0.728527067842925\\
627	1.92741364670176\\
628	2.76126966263812\\
629	2.32218416126096\\
630	0.823476352969465\\
631	0.757600741126276\\
632	1.13469399302578\\
633	0.512119633385463\\
634	0.446400493246496\\
635	0.75836182055086\\
636	0.589067160630994\\
637	0.37905794392298\\
638	0.316812474253128\\
639	0.305450964733822\\
640	0.411043581989764\\
641	0.427106762218403\\
642	0.328250113650151\\
643	0.26924910437628\\
644	0.369608682235869\\
645	0.545590263146875\\
646	0.51095126624569\\
647	0.225969068566234\\
648	0.0937504933582185\\
649	0.155691757863386\\
650	0.143563257569863\\
651	0.351751358524464\\
652	0.433949888254632\\
653	0.357377637035286\\
654	0.214611614026436\\
655	0.149150874346242\\
656	0.307133618961262\\
657	0.445991544391623\\
658	0.446780348229075\\
659	0.316092954418524\\
660	0.237936921677401\\
661	0.331447133374475\\
662	0.401124318471257\\
663	0.429152751370368\\
664	0.489548512639156\\
665	0.558465944697459\\
666	0.557289567781402\\
667	0.435843112962036\\
668	0.287024219880359\\
669	0.355205661752089\\
670	0.497515711480277\\
671	0.531266933596987\\
672	0.511681944710538\\
673	0.464540147486314\\
674	0.332596060914124\\
675	0.158226012455663\\
676	0.014064244121227\\
677	0.16322630420039\\
678	0.256088197379928\\
679	0.140405968224649\\
680	0.138933251088646\\
681	0.362448154041183\\
682	0.328641526594172\\
683	0.0768127486016245\\
684	0.384845493789722\\
685	0.482809793757096\\
686	0.275457514423422\\
687	0.0530417323123215\\
688	0.268408350840961\\
689	0.299749000323586\\
690	0.283898391991072\\
691	0.31595235382063\\
692	0.325186823438073\\
693	0.364058648683932\\
694	0.421687748481671\\
695	0.358488699794597\\
696	0.141064270962358\\
697	0.123110445853515\\
698	0.223346033159042\\
699	0.237056305046239\\
700	0.351137018832305\\
701	0.420089415453384\\
702	0.38676276652493\\
703	0.342165439154439\\
704	0.257617649191544\\
705	0.129631691943858\\
706	0.102069193589967\\
707	0.180578295632904\\
708	0.301444002698921\\
709	0.336558240944508\\
710	0.281824011800399\\
711	0.286293675127396\\
712	0.402233019751339\\
713	0.523937042275589\\
714	0.588326894756645\\
715	0.636902666331705\\
716	0.617199255461832\\
717	0.412922613399484\\
718	0.156349702270328\\
719	0.117043838122135\\
720	0.0829060412724096\\
721	0.117524377907638\\
722	0.162973175072084\\
723	0.153400691364329\\
724	0.155761355919494\\
725	0.136505908666391\\
726	0.171157307898558\\
727	0.315735424875619\\
728	0.40461518778541\\
729	0.303507250567475\\
730	0.0483486889728578\\
731	0.260672915757872\\
732	0.4378087280427\\
733	0.377545678447735\\
734	0.249940420286451\\
735	0.230510124707139\\
736	0.0807542714077559\\
737	0.223955075594841\\
738	0.417924833504463\\
739	0.382960190580908\\
740	0.230991749352918\\
741	0.217866072182273\\
742	0.259138390261212\\
743	0.15926043649283\\
744	0.160276929577583\\
745	0.368243891262315\\
746	0.266660499707543\\
747	0.0867255979748171\\
748	0.308207283774522\\
749	0.337854629420402\\
750	0.392685997932947\\
751	0.248407149353968\\
752	0.110689032013529\\
753	0.100710789577973\\
754	0.892288182394515\\
755	2.09382446624789\\
756	2.72470634861384\\
757	2.27675888323154\\
758	1.1048354005623\\
759	0.226444553580792\\
760	0.369864784351082\\
761	0.148496302986251\\
762	0.365909290815374\\
763	0.578604806929237\\
764	0.654476355668374\\
765	0.687499399988648\\
766	0.567210356681008\\
767	0.301658546688265\\
768	0.0861310867565108\\
769	0.14846355624387\\
770	0.0995700847362622\\
771	0.143875236501039\\
772	0.389489306782707\\
773	0.417962803226999\\
774	0.230849392667428\\
775	0.361682033254571\\
776	0.583304032543313\\
777	0.615267969742941\\
778	0.45043856213961\\
779	0.183215266958946\\
780	0.294618630234423\\
781	0.367472403217824\\
782	0.28867754268478\\
783	0.268665809606939\\
784	0.310565815729897\\
785	0.189865358428593\\
786	0.113698506102315\\
787	0.366620280575616\\
788	0.454536407047754\\
789	0.372589715719288\\
790	0.241980359965275\\
791	0.34976453245567\\
792	0.662577186432819\\
793	0.841524117152697\\
794	0.716005275127444\\
795	0.331030214489267\\
796	0.250144896561747\\
797	0.449989176152152\\
798	0.327684151304352\\
799	0.123866547932481\\
800	0.326521572234744\\
801	0.470510345760954\\
802	0.585410715276073\\
803	0.632918141515959\\
804	0.770145897070932\\
805	1.0016042480738\\
806	0.963117036501962\\
807	0.514322799162361\\
808	0.37323405525437\\
809	0.597743450479532\\
810	0.315490613586239\\
811	0.208054317321593\\
812	0.462301152403848\\
813	0.255827124984059\\
814	0.463020198255113\\
815	0.944596262342304\\
816	1.11759538454457\\
817	1.01220163116186\\
818	0.78900310133871\\
819	0.487556736437165\\
820	0.44301664118515\\
821	0.634236253453442\\
822	0.357557033583092\\
823	0.32850876953164\\
824	0.866194518175319\\
825	0.845466292276557\\
826	0.338485526354025\\
827	0.235440966410065\\
828	0.333226140420261\\
829	0.200646713494725\\
830	0.340872647510832\\
831	0.270273983035593\\
832	0.192448921722253\\
833	0.484896293904654\\
834	0.451977294820869\\
835	0.2849984536338\\
836	0.717342889084254\\
837	1.03050679519872\\
838	0.946487938823356\\
839	0.481260957218292\\
840	0.145304001353818\\
841	0.493649897983945\\
842	0.585434107017717\\
843	0.568750458513584\\
844	0.578379026679306\\
845	0.677112466134451\\
846	0.773487569592078\\
847	0.678688970458506\\
848	0.421757035797303\\
849	0.2010576960115\\
850	0.19277736270908\\
851	0.277599395814113\\
852	0.443628705738444\\
853	0.621054613271224\\
854	0.597393241573949\\
855	0.379979473497789\\
856	0.264784173047269\\
857	0.314257544445957\\
858	0.315291996084775\\
859	0.17725450429281\\
860	0.251416215795789\\
861	0.458076562431085\\
862	0.482037570301275\\
863	0.311524892391306\\
864	0.406386224835557\\
865	0.700325161227682\\
866	0.90092444867141\\
867	0.931777201215463\\
868	0.730868391450711\\
869	0.382514810859039\\
870	0.41352761570972\\
871	0.627894590807779\\
872	0.617899435559013\\
873	0.64207227087695\\
874	0.72886369486967\\
875	0.598596620180397\\
876	0.318310803702694\\
877	0.131709128367549\\
878	0.118179883343233\\
879	0.189556078563537\\
880	0.17781323599985\\
881	0.389564363991561\\
882	0.672205993409351\\
883	0.669718091752701\\
884	0.369472221493465\\
885	0.211527861038651\\
886	0.359648591242743\\
887	0.335095273956787\\
888	0.068483572010832\\
889	0.353621066309294\\
890	0.568877644257586\\
891	0.512116123644591\\
892	0.363679148902602\\
893	0.288405450400304\\
894	0.22326926166577\\
895	0.176885819615737\\
896	0.143154711145315\\
897	0.380298239514034\\
898	0.614558038215729\\
899	0.578721040014523\\
900	0.21139161194512\\
901	0.478174607734529\\
902	0.926275928433847\\
903	0.997289257818208\\
904	0.77947958085551\\
905	0.576055322186855\\
906	0.49035681801128\\
907	0.460677476901321\\
908	0.386568329360941\\
909	0.209387809400451\\
910	0.200327469427142\\
911	0.230691513087918\\
912	0.192932952506158\\
913	0.503266993024592\\
914	0.755507328596432\\
915	0.809045996551424\\
916	0.767021286444002\\
917	0.671065960090064\\
918	0.484883129151955\\
919	0.230074327994701\\
920	0.128995454838941\\
921	0.345289479117791\\
922	0.384530814718324\\
923	0.292102302697931\\
924	0.298194920185229\\
925	0.35771921663781\\
926	0.394110421937291\\
927	0.431825529604142\\
928	0.553954180352857\\
929	0.716574681295351\\
930	0.676863649329791\\
931	0.339434413748892\\
932	0.284926107480752\\
933	0.536176706352768\\
934	0.409485156296245\\
935	0.0762554688483868\\
936	0.381893559603867\\
937	0.591039664532489\\
938	0.717101620199821\\
939	0.744777453233646\\
940	0.597789905799106\\
941	0.354750384906439\\
942	0.141721272283775\\
943	0.0911511701068049\\
944	0.324441072364926\\
945	0.461859561931817\\
946	0.511780721756483\\
947	0.444893522509152\\
948	0.234716421721703\\
949	0.186950925756396\\
950	0.298056607842396\\
951	0.405147744830547\\
952	0.615414575875029\\
953	0.770303101705183\\
954	0.661038925155761\\
955	0.239378433920133\\
956	0.365514705120487\\
957	0.67376659181255\\
958	0.547328374349141\\
959	0.305730004524237\\
960	0.503619749928028\\
961	0.585707110598313\\
962	0.459221706332411\\
963	0.294895170063773\\
964	0.488273674110233\\
965	0.745203734028448\\
966	0.64626480370085\\
967	0.264948406022545\\
968	0.361722366213968\\
969	0.486496333188186\\
970	0.295219887811328\\
971	0.129587843218795\\
972	0.285823624471815\\
973	0.451867922940138\\
974	0.575316066695998\\
975	0.509671043852899\\
976	0.417299228166569\\
977	0.496436062428654\\
978	0.403912769597126\\
979	0.404639763167882\\
980	0.695908585816227\\
981	0.650959266534144\\
982	0.246281558474422\\
983	0.305011672241064\\
984	0.441007638268125\\
985	0.259551396929417\\
986	0.165688273042358\\
987	0.241582589764807\\
988	0.371211893805178\\
989	0.546969484095499\\
990	0.424203832364138\\
991	0.0452383346885032\\
992	0.585255029354417\\
993	0.909120952621958\\
994	0.946811307020979\\
995	0.788343965788801\\
996	0.386546193458253\\
997	0.239025539584588\\
998	0.642674117585814\\
999	0.606721730003367\\
1000	0.592404430132469\\
1001	1.03245015558821\\
1002	1.20503527281041\\
1003	0.928624688851519\\
1004	0.498314311077874\\
1005	0.349173572074039\\
1006	0.607183413147065\\
1007	0.763295105733744\\
1008	0.570345455987628\\
1009	0.21093011888463\\
1010	0.215819881142273\\
1011	0.144461284832156\\
1012	0.283619701229973\\
1013	0.560104217293436\\
1014	0.444574345032972\\
1015	0.346305885930186\\
1016	0.794927016549784\\
1017	0.904694076731957\\
1018	0.668544506897595\\
1019	0.528020390528739\\
1020	0.477734566047601\\
1021	0.227119071141285\\
1022	0.0925806252615383\\
1023	0.25067287828687\\
1024	0.283187457148412\\
1025	0.379950655052656\\
1026	0.49112718634571\\
1027	0.484495701019795\\
1028	0.363910694066386\\
1029	0.253655544960424\\
1030	0.349967028147319\\
1031	0.475196603542184\\
1032	0.426949660516203\\
1033	0.169082090744189\\
1034	0.162119757585455\\
1035	0.339376359084902\\
1036	0.259816193941616\\
1037	0.0412933281173994\\
1038	0.185584914987884\\
1039	0.320377974683441\\
1040	0.334370156070599\\
1041	0.33331436860529\\
1042	0.53520542233906\\
1043	0.769969398221327\\
1044	0.854821529024577\\
1045	0.800875856051569\\
1046	0.687423797502815\\
1047	0.545289680196176\\
1048	0.42167654154754\\
1049	0.42841287148455\\
1050	0.457620099842055\\
1051	0.372898849396488\\
1052	0.435698731565404\\
1053	0.522148097710112\\
1054	0.46973975231359\\
1055	0.522808385813571\\
1056	0.598475629155794\\
1057	0.526006123640873\\
1058	0.378789260486091\\
1059	0.140254970361527\\
1060	0.309262419596818\\
1061	0.559509266463221\\
1062	0.444685801464865\\
1063	0.180903613123284\\
1064	0.375480830790108\\
1065	0.486759948062772\\
1066	0.528990062898542\\
1067	0.690590499514008\\
1068	0.825094891624801\\
1069	0.734675746313926\\
1070	0.477332325473237\\
1071	0.217382606934343\\
1072	0.135171217567511\\
1073	0.279336680480251\\
1074	0.445957842365257\\
1075	0.460734973751249\\
1076	0.34985151711341\\
1077	0.303178153572367\\
1078	0.311144788015777\\
1079	0.262159179469428\\
1080	0.228502472370588\\
1081	0.226380896291226\\
1082	0.281467948209267\\
1083	0.557433535699452\\
1084	0.67183231063932\\
1085	0.375671449583628\\
1086	0.365741672063115\\
1087	0.766083911967129\\
1088	0.720087427482685\\
1089	0.380953990653822\\
1090	0.106769914684017\\
1091	0.241901772157984\\
1092	0.537688374778057\\
1093	0.650568738555171\\
1094	0.536890085191081\\
1095	0.455555163306814\\
1096	0.58621486718669\\
1097	0.707034653989684\\
1098	0.65530032417563\\
1099	0.489747111200401\\
1100	0.338151122593478\\
1101	0.251979898003424\\
1102	0.135855279159459\\
1103	0.215803588864905\\
1104	0.650360464681284\\
1105	0.882057484209142\\
1106	0.70019725987026\\
1107	0.353104412148447\\
1108	0.256587715738428\\
1109	0.237288037810344\\
1110	0.393964746137633\\
1111	0.476217075736521\\
1112	0.318033905928838\\
1113	0.443879095276938\\
1114	0.644794411414292\\
1115	0.501330978557401\\
1116	0.34218826846042\\
1117	0.59975282667751\\
1118	0.639671175083582\\
1119	0.529440269571555\\
1120	0.442703961150953\\
1121	0.265571595743326\\
1122	0.0689799097070581\\
1123	0.0295184044071393\\
1124	0.302260497346136\\
1125	0.723842675428941\\
1126	1.03098294540808\\
1127	1.01794838350118\\
1128	0.69036817228119\\
1129	0.356142128195849\\
1130	0.251905649516953\\
1131	0.0457731851139892\\
1132	0.254567692822892\\
1133	0.3688442612567\\
1134	0.236323872144101\\
1135	0.0679040881632075\\
1136	0.125461525072873\\
1137	0.469010299655276\\
1138	0.765268684112713\\
1139	0.605334258081494\\
1140	0.110805941330764\\
1141	0.65413473645282\\
1142	0.775929830163297\\
1143	0.33783528857876\\
1144	0.499959376291521\\
1145	0.991085767405238\\
1146	1.04473372207031\\
1147	0.862493951239157\\
1148	0.828939012039119\\
1149	0.700123345347914\\
1150	0.328977766369524\\
1151	0.472657154435168\\
1152	0.773172140703431\\
1153	0.697150003185472\\
1154	0.378670664484028\\
1155	0.221105301360738\\
1156	0.383486203379865\\
1157	0.545016184823472\\
1158	0.511244904370802\\
1159	0.39227687059048\\
1160	0.460807570279434\\
1161	0.439523697281284\\
1162	0.291494522656188\\
1163	0.383684232703596\\
1164	0.519634197540048\\
1165	0.61932903215293\\
1166	0.761265289935176\\
1167	0.803016168115657\\
1168	0.647991605263647\\
1169	0.602866526857985\\
1170	0.773911131826815\\
1171	0.736470442948859\\
1172	0.572068499897697\\
1173	0.58688891169514\\
1174	0.53193265149407\\
1175	0.295734220097046\\
1176	0.145461313482281\\
1177	0.0991807112175916\\
1178	0.224273205654516\\
1179	0.478316398035087\\
1180	0.559783638091041\\
1181	0.356184834686581\\
1182	0.0453274277599695\\
1183	0.343521703226387\\
1184	0.270827194586928\\
1185	0.162427060387293\\
1186	0.534254838013541\\
1187	0.640866133262533\\
1188	0.462657131574164\\
1189	0.172418666693398\\
1190	0.224854468024476\\
1191	0.418669374328701\\
1192	0.488612052152037\\
1193	0.588817912376539\\
1194	0.77360374248751\\
1195	0.812018551301868\\
1196	0.658352789595154\\
1197	0.390596231400963\\
1198	0.23827148136614\\
1199	0.404827095768551\\
1200	0.458691478726506\\
1201	0.300815515886928\\
1202	0.137324499370226\\
1203	0.192771691782227\\
1204	0.197961361748826\\
1205	0.262875959586602\\
1206	0.513629515085065\\
1207	0.658929408589692\\
1208	0.62628468586425\\
1209	0.401864753268883\\
1210	0.418855399059409\\
1211	0.929971335270184\\
1212	1.09304835565284\\
1213	0.647408845547438\\
1214	0.146694813090288\\
1215	0.603089680390987\\
1216	0.5841488546429\\
1217	0.333378120106976\\
1218	0.34985671638862\\
1219	0.299355121876421\\
1220	0.407219086419217\\
1221	0.605232209341661\\
1222	0.592956925612315\\
1223	0.532856920497668\\
1224	0.59106296842982\\
1225	0.61984232372637\\
1226	0.520755847761676\\
1227	0.338911192911197\\
1228	0.214643970178203\\
1229	0.332313998414965\\
1230	0.557451650066823\\
1231	0.714965617649123\\
1232	0.693508699799119\\
1233	0.497820549883622\\
1234	0.229763244220124\\
1235	0.116698895862632\\
1236	0.143712139309626\\
1237	0.0626835357143963\\
1238	0.171615001161818\\
1239	0.224652994070805\\
1240	0.157032787671615\\
1241	0.223628704389018\\
1242	0.268512572569072\\
1243	0.109916879148358\\
1244	0.183170719689723\\
1245	0.202630051511414\\
1246	0.271150301827152\\
1247	0.550903828648659\\
1248	0.556520755988829\\
1249	0.184023952889716\\
1250	0.325121751144312\\
1251	0.525034679801723\\
1252	0.446567720983772\\
1253	0.43986823241786\\
1254	0.360307583004613\\
1255	0.291917748925662\\
1256	0.526778102163278\\
1257	0.473602111066009\\
1258	0.150203972358194\\
1259	0.498038640805559\\
1260	0.719681661404805\\
1261	0.530355745447569\\
1262	0.132779501378234\\
1263	0.547625044845395\\
1264	0.87262052544942\\
1265	0.845448945970716\\
1266	0.570558607969083\\
1267	0.30716266192415\\
1268	0.224379441038601\\
1269	0.256888346326299\\
1270	0.281433419924608\\
1271	0.327782452239766\\
1272	0.233830663408589\\
1273	0.259773287580726\\
1274	0.708583714735372\\
1275	0.951285282807045\\
1276	0.881864177328423\\
1277	0.70344340414887\\
1278	0.486100309189704\\
1279	0.428583390388548\\
1280	0.664360069128585\\
1281	0.607372433909361\\
1282	0.430946045052881\\
1283	0.546728475566623\\
1284	0.393762968204706\\
1285	0.231078260967565\\
1286	0.573098363495476\\
1287	0.434542295305657\\
1288	0.352371377285696\\
1289	0.747537064510239\\
1290	0.610621102269335\\
1291	0.0747920167346156\\
1292	0.475832030844646\\
1293	0.410130751725635\\
1294	0.16547996611665\\
1295	0.578930763884711\\
1296	0.651049540373431\\
1297	0.531029451179755\\
1298	0.432366339840194\\
1299	0.130185243981183\\
1300	0.31726363500407\\
1301	0.480090960913724\\
1302	0.226926168568983\\
1303	0.36601594436566\\
1304	0.630906682990481\\
1305	0.492293528475987\\
1306	0.160454502598636\\
1307	0.256593594491305\\
1308	0.160818566535714\\
1309	0.242753384299684\\
1310	0.437269131423912\\
1311	0.421284028374232\\
1312	0.37973116848521\\
1313	0.226902995070559\\
1314	0.152660218731496\\
1315	0.503422345959693\\
1316	0.577248098355658\\
1317	0.640724226041516\\
1318	0.796339264758414\\
1319	0.57352494397008\\
1320	0.130246140716947\\
1321	0.741283678916919\\
1322	0.859313882981998\\
1323	0.389706361576307\\
1324	0.340196487812669\\
1325	0.770704054540422\\
1326	0.837866279533662\\
1327	0.935400134899361\\
1328	1.03905743039519\\
1329	0.795925771530005\\
1330	0.468032723565019\\
1331	0.539873684392969\\
1332	0.605687510681739\\
1333	0.500576545780843\\
1334	0.346263654443717\\
1335	0.259816805008787\\
1336	0.664034475052776\\
1337	0.968595214884648\\
1338	0.8824804880205\\
1339	0.533724640621101\\
1340	0.227783428249371\\
1341	0.20249437737224\\
1342	0.293130549517616\\
1343	0.159046050217638\\
1344	0.36377413233226\\
1345	0.69752114031682\\
1346	0.652099903024062\\
1347	0.307334395570351\\
1348	0.223854651493835\\
1349	0.30168208871521\\
1350	0.207828604479985\\
1351	0.343799698865816\\
1352	0.533610839914502\\
1353	0.70556680256432\\
1354	0.863192795087905\\
1355	0.831737747556242\\
1356	0.631907029087104\\
1357	0.5137763528493\\
1358	0.585196099286018\\
1359	0.613702054817861\\
1360	0.602936153783279\\
1361	0.567399964914442\\
1362	0.589320445954468\\
1363	0.764057654163899\\
1364	0.803958595123603\\
1365	0.579186512191239\\
1366	0.25996479291888\\
1367	0.122498625348993\\
1368	0.155464229200524\\
1369	0.33165550761421\\
1370	0.565820310402731\\
1371	0.67158584363341\\
1372	0.52862799172766\\
1373	0.18009410056957\\
1374	0.170094383848948\\
1375	0.358153528480492\\
1376	0.416830070706143\\
1377	0.388002202757937\\
1378	0.333692190705088\\
1379	0.577876406791921\\
1380	0.781234780537293\\
1381	0.590130036925739\\
1382	0.215144485880523\\
1383	0.377997633866649\\
1384	0.334165715505774\\
1385	0.347712491684665\\
1386	0.605494306550604\\
1387	0.558779162658729\\
1388	0.275030291429447\\
1389	0.438048885110263\\
1390	0.615210082588837\\
1391	0.470376200099858\\
1392	0.075376165123468\\
1393	0.317479368063406\\
1394	0.460959272202355\\
1395	0.289061590065403\\
1396	0.0965547109431708\\
1397	0.429068038699018\\
1398	0.684747053717749\\
1399	0.802095111877916\\
1400	0.660591680312876\\
1401	0.284518302401025\\
1402	0.112793071141255\\
1403	0.252839929596487\\
1404	0.153441295287371\\
1405	0.0936919140959101\\
1406	0.164971311747485\\
1407	0.155397031489008\\
1408	0.115285135714112\\
1409	0.0340923476570214\\
1410	0.0799583256368405\\
1411	0.164281044903229\\
1412	0.217778476425612\\
1413	0.236353930910212\\
1414	0.237218514936038\\
1415	0.430002868388864\\
1416	0.637022472259674\\
1417	0.57019516048719\\
1418	0.224922262017431\\
1419	0.291774544804939\\
1420	0.506593593206139\\
1421	0.440514998058237\\
1422	0.279382172276462\\
1423	0.211445482168005\\
1424	0.293144781968962\\
1425	0.384622427109352\\
1426	0.238848945352705\\
1427	0.147317755794344\\
1428	0.409681653271499\\
1429	0.328744534530124\\
1430	0.307495569650739\\
1431	0.577393295878819\\
1432	0.542850189786511\\
1433	0.365796175090139\\
1434	0.513229149573472\\
1435	0.534608164456783\\
1436	0.495918766262856\\
1437	0.668277855201852\\
1438	0.744073976097762\\
1439	0.764702737189493\\
1440	0.934586837577323\\
1441	1.03812218248735\\
1442	0.836191214565965\\
1443	0.464258566959064\\
1444	0.169535901094871\\
1445	0.113992130113522\\
1446	0.254808324581197\\
1447	0.49980735527322\\
1448	0.748036409192327\\
1449	0.851605365317003\\
1450	0.762001275749619\\
1451	0.602776861356764\\
1452	0.417481710675782\\
1453	0.321747881724357\\
1454	0.424646541446106\\
1455	0.379908396369887\\
1456	0.161342177031815\\
1457	0.19801631102434\\
1458	0.299799029235775\\
1459	0.262251885965337\\
1460	0.0887279057633941\\
1461	0.108515857436239\\
1462	0.173849196024466\\
1463	0.157468459263039\\
1464	0.337071186146531\\
1465	0.536626375916972\\
1466	0.553205772905676\\
1467	0.347342421089263\\
1468	0.19724716152145\\
1469	0.494271554916131\\
1470	0.698499306279125\\
1471	0.691930691183403\\
1472	0.606375629859606\\
1473	0.707518250906747\\
1474	0.824616213688261\\
1475	0.793500224582173\\
1476	0.729366637033635\\
1477	0.664611738597242\\
1478	0.52370230940254\\
1479	0.333514585471706\\
1480	0.238534216177588\\
1481	0.306084159235254\\
1482	0.424997771970866\\
1483	0.505791990487111\\
1484	0.750215597554343\\
1485	0.961646143370971\\
1486	0.783346529075394\\
1487	0.371102482964568\\
1488	0.265176950414104\\
1489	0.194714110789823\\
1490	0.382257204936566\\
1491	0.498624847956222\\
1492	0.304282106676666\\
1493	0.0294748766772345\\
1494	0.11657125649192\\
1495	0.212232346605573\\
1496	0.210276142287185\\
1497	0.164668224241788\\
1498	0.656823222786021\\
1499	1.09572279941734\\
1500	1.23271820206479\\
1501	1.02382180752791\\
1502	0.657754359733899\\
1503	0.356484330115183\\
1504	0.15590374047752\\
1505	0.187342635823333\\
1506	0.241514321563142\\
1507	0.396893267721729\\
1508	0.604339451345821\\
1509	0.655254102938723\\
1510	0.542009433502955\\
1511	0.469100638897761\\
1512	0.453113492760237\\
1513	0.314607514309541\\
1514	0.137343416317913\\
1515	0.16826299140548\\
1516	0.181742377470477\\
1517	0.182797846910563\\
1518	0.308689656918434\\
1519	0.589123323601346\\
1520	0.835313388164985\\
1521	0.751629003459079\\
1522	0.425765689694949\\
1523	0.547828969070655\\
1524	0.603914655637977\\
1525	0.15056488390933\\
1526	0.679036644064575\\
1527	1.14609056122336\\
1528	1.01815139649352\\
1529	0.528707778915703\\
1530	0.456121770035782\\
1531	0.637418301682629\\
1532	0.604897211403683\\
1533	0.494209605889778\\
1534	0.392914525483137\\
1535	0.291254055925102\\
1536	0.18767357227049\\
1537	0.0207076900994457\\
1538	0.238374902563699\\
1539	0.474789185025303\\
1540	0.583533472171029\\
1541	0.404006993570605\\
1542	0.203710792284998\\
1543	0.605502417188802\\
1544	0.704593915047504\\
1545	0.444880526841348\\
1546	0.279093387169363\\
1547	0.292639014473523\\
1548	0.215049338488645\\
1549	0.270326950837429\\
1550	0.244801524994023\\
1551	0.181510477580052\\
1552	0.224869918924013\\
1553	0.271706678252936\\
1554	0.274144619033362\\
1555	0.17580625686758\\
1556	0.242694444525501\\
1557	0.258397950526723\\
1558	0.0956599162969102\\
1559	0.290222185569757\\
1560	0.351433561121964\\
1561	0.300394374565569\\
1562	0.479265168397394\\
1563	0.649328211190724\\
1564	0.634663383590059\\
1565	0.547333661679593\\
1566	0.494284355040153\\
1567	0.494290642689591\\
1568	0.565686580797289\\
1569	0.560437415132723\\
1570	0.424243327881649\\
1571	0.345848926254892\\
1572	0.438231643285417\\
1573	0.615049252673882\\
1574	0.701390423424249\\
1575	0.622700803693321\\
1576	0.674977636619207\\
1577	0.953149809404131\\
1578	1.0393326994093\\
1579	0.849832443716062\\
1580	0.502794951175903\\
1581	0.211727202939577\\
1582	0.32342952961564\\
1583	0.506574826803359\\
1584	0.627119819372067\\
1585	0.642723340756867\\
1586	0.641888180423328\\
1587	0.756943166760529\\
1588	0.84814669773893\\
1589	0.68161657518207\\
1590	0.32363378714536\\
1591	0.32192449704191\\
1592	0.501449757455932\\
1593	0.435855020329214\\
1594	0.34711203952262\\
1595	0.443056135385426\\
1596	0.444965646273248\\
1597	0.300563784592226\\
1598	0.0764264434798999\\
1599	0.181695904004665\\
1600	0.334579393179548\\
1601	0.319044405760889\\
1602	0.445031691929692\\
1603	0.629429890692383\\
1604	0.493733679381599\\
1605	0.152155005505426\\
1606	0.0759928109827516\\
1607	0.0914807713113918\\
1608	0.295623186276377\\
1609	0.380502247066614\\
1610	0.278497491019325\\
1611	0.41292227879425\\
1612	0.655171716295165\\
1613	0.775315486282599\\
1614	0.696294952709069\\
1615	0.433711974700216\\
1616	0.0862681763965909\\
1617	0.376883810257876\\
1618	0.673774013762568\\
1619	0.733483198095749\\
1620	0.551110686876951\\
1621	0.334866439383021\\
1622	0.335490053758092\\
1623	0.282014204821392\\
1624	0.204297074105571\\
1625	0.292157906209206\\
1626	0.359948300288946\\
1627	0.331901332884775\\
1628	0.255948939125909\\
1629	0.255151019654955\\
1630	0.266720614039642\\
1631	0.168961479149183\\
1632	0.179284972580033\\
1633	0.382252994018345\\
1634	0.427137736627872\\
1635	0.319231601711931\\
1636	0.585044016277411\\
1637	0.841536394691967\\
1638	0.754912513932417\\
1639	0.415081467743279\\
1640	0.199399019259145\\
1641	0.392068873063899\\
1642	0.611797588998442\\
1643	0.680657938972765\\
1644	0.563426494184934\\
1645	0.787709302015365\\
1646	1.20594359538766\\
1647	1.29837578605631\\
1648	0.998701783130613\\
1649	0.715611903207284\\
1650	0.588676241494111\\
1651	0.279792462229601\\
1652	0.528165602933636\\
1653	0.800935453727805\\
1654	0.714922357792839\\
1655	0.627488151798378\\
1656	0.663887348016648\\
1657	0.497836241079376\\
1658	0.327088688908173\\
1659	0.347322188750708\\
1660	0.238329328289809\\
1661	0.0712864695448951\\
1662	0.379588755149812\\
1663	0.483105518939034\\
1664	0.303578630991329\\
1665	0.206145496176095\\
1666	0.401980103987141\\
1667	0.287190020018585\\
1668	0.105419643485365\\
1669	0.466999461841536\\
1670	0.540043817356454\\
1671	0.400177355704686\\
1672	0.359828155374772\\
1673	0.433206439170569\\
1674	0.54323097556504\\
1675	0.60751248011818\\
1676	0.590795541352951\\
1677	0.546809246778224\\
1678	0.5169894031549\\
1679	0.413795733187278\\
1680	0.166765951294089\\
1681	0.148878372014209\\
1682	0.224984937243948\\
1683	0.188444103082586\\
1684	0.409156725507385\\
1685	0.511658960979729\\
1686	0.532633195126255\\
1687	0.719299166955618\\
1688	0.830567382328121\\
1689	0.677732142356846\\
1690	0.338113815824965\\
1691	0.00864796147419466\\
1692	0.178513036165022\\
1693	0.264815414319966\\
1694	0.431297379954816\\
1695	0.537636733031649\\
1696	0.528537937856241\\
1697	0.484294479402469\\
1698	0.429906035094518\\
1699	0.420193200253821\\
1700	0.489625022718731\\
1701	0.410818966893969\\
1702	0.144009561097052\\
1703	0.409040468083817\\
1704	0.514610200631623\\
1705	0.380100524103092\\
1706	0.632816129130505\\
1707	0.865802787218027\\
1708	0.658942182772927\\
1709	0.17332413720837\\
1710	0.430888467321637\\
1711	0.620386895143715\\
1712	0.731575804065528\\
1713	0.783684272381219\\
1714	0.602088489365523\\
1715	0.249224913746784\\
1716	0.188812519764418\\
1717	0.328894448080919\\
1718	0.347852006080635\\
1719	0.325963981818339\\
1720	0.363001156143509\\
1721	0.4444449199134\\
1722	0.380202400768175\\
1723	0.070289721244262\\
1724	0.441836250245204\\
1725	0.643757758342175\\
1726	0.46803402986558\\
1727	0.166031957412227\\
1728	0.45395808451906\\
1729	0.702141426244159\\
1730	0.703784407490947\\
1731	0.495285724817464\\
1732	0.229362470829893\\
1733	0.18018516447027\\
1734	0.318944384864367\\
1735	0.317927445283448\\
1736	0.46054154051984\\
1737	0.702589027811511\\
1738	0.683943442583302\\
1739	0.373775816084005\\
1740	0.169722520634252\\
1741	0.338315240238975\\
1742	0.144333870543602\\
1743	0.440384307761278\\
1744	0.814420358699666\\
1745	0.718483647597573\\
1746	0.32268731585597\\
1747	0.359864963641841\\
1748	0.207848986539807\\
1749	0.345323283707884\\
1750	0.718899145383002\\
1751	0.602093901217059\\
1752	0.14298844148785\\
1753	0.220527145071217\\
1754	0.32640146345477\\
1755	0.521787416108978\\
1756	0.651294591329915\\
1757	0.587014598532468\\
1758	0.526052230201919\\
1759	0.573603752973317\\
1760	0.680416753269935\\
1761	0.732169462452854\\
1762	0.680776645788326\\
1763	0.627892728867725\\
1764	0.503636633987628\\
1765	0.211073442047971\\
1766	0.106748562463938\\
1767	0.2519104558373\\
1768	0.249547372212585\\
1769	0.308564718556851\\
1770	0.371665881955158\\
1771	0.215738701813582\\
1772	0.129847039022412\\
1773	0.318679907180183\\
1774	0.192558895691456\\
1775	0.219297589936507\\
1776	0.586573491221457\\
1777	0.666228111212966\\
1778	0.523869784327977\\
1779	0.374741248077429\\
1780	0.243525301902685\\
1781	0.3114380934455\\
1782	0.409703985961431\\
1783	0.521656342436072\\
1784	0.777787192321129\\
1785	0.894272496841148\\
1786	0.726396159372864\\
1787	0.424452791460504\\
1788	0.303573113318997\\
1789	0.414723850765126\\
1790	0.498864057096193\\
1791	0.482815293125504\\
1792	0.294890405512635\\
1793	0.248904065020524\\
1794	0.575710431415076\\
1795	0.636076427713717\\
1796	0.442446388957629\\
1797	0.443854020601148\\
1798	0.523214396256606\\
1799	0.425124359277182\\
1800	0.201744379056123\\
1801	0.232109527902345\\
1802	0.414974456144079\\
1803	0.474769246837259\\
1804	0.481632610438582\\
1805	0.377076821374018\\
1806	0.32677729387552\\
1807	0.391087274691477\\
1808	0.159493085451673\\
1809	0.337192868726371\\
1810	0.633087865580532\\
1811	0.561651065938799\\
1812	0.321369064292441\\
1813	0.15161088652274\\
1814	0.206574082764142\\
1815	0.447062249551755\\
1816	0.610215060409397\\
1817	0.648215897876746\\
1818	0.674025408997679\\
1819	0.729340456211708\\
1820	0.698660790594277\\
1821	0.541160782392076\\
1822	0.466185921581845\\
1823	0.550623649997345\\
1824	0.606986123731155\\
1825	0.516171265549133\\
1826	0.345226442558457\\
1827	0.248191179778469\\
1828	0.0892266998795086\\
1829	0.433909341962317\\
1830	0.754318129309783\\
1831	0.664431641323019\\
1832	0.329225829178854\\
1833	0.655409375235393\\
1834	0.729642090549517\\
1835	0.382752268731833\\
1836	0.656446849303817\\
1837	0.967394567844947\\
1838	0.900884884647638\\
1839	0.815477882057038\\
1840	0.792280007705258\\
1841	0.715799514332808\\
1842	0.683125675196396\\
1843	0.691752660208801\\
1844	0.572318506740686\\
1845	0.268882608027113\\
1846	0.20265260884146\\
1847	0.537353796860442\\
1848	0.675935613686201\\
1849	0.547771642363726\\
1850	0.143182021207272\\
1851	0.402145168062323\\
1852	0.690501532912754\\
1853	0.576053182123254\\
1854	0.284082162876035\\
1855	0.334521188018458\\
1856	0.142370838658672\\
1857	0.42363827390215\\
1858	0.854454957975878\\
1859	0.798287821589539\\
1860	0.624514351774396\\
1861	0.906409172067576\\
1862	1.00226550734445\\
1863	0.749606569771726\\
1864	0.50371736167051\\
1865	0.465389713935086\\
1866	0.532793809914006\\
1867	0.543473276857849\\
1868	0.469212063806816\\
1869	0.459104715646875\\
1870	0.440760921931601\\
1871	0.293198970709976\\
1872	0.133154622443362\\
1873	0.0872376709119251\\
1874	0.18130100745058\\
1875	0.249879819440566\\
1876	0.241986487424363\\
1877	0.259862040075429\\
1878	0.315833362768093\\
1879	0.324173875089168\\
1880	0.354927558284102\\
1881	0.332242780649641\\
1882	0.152313127279675\\
1883	0.087915981429259\\
1884	0.249263379638627\\
1885	0.272092125327508\\
1886	0.177073618935148\\
1887	0.0550400321095974\\
1888	0.0460076618309449\\
1889	0.163245750685168\\
1890	0.372961458958888\\
1891	0.417133398283721\\
1892	0.127048350914136\\
1893	0.344229035256574\\
1894	0.579277843719862\\
1895	0.391657446652552\\
1896	0.312962950975955\\
1897	0.580727781734763\\
1898	0.696142656221254\\
1899	0.798543596928025\\
1900	0.69980281023454\\
1901	0.284395466561834\\
1902	0.155209643675847\\
1903	0.189784908146297\\
1904	0.155176894581428\\
1905	0.441540453544867\\
1906	0.479876166554192\\
1907	0.304162023366081\\
1908	0.119011862119111\\
1909	0.0927537325081903\\
1910	0.130351550387097\\
1911	0.26216560964769\\
1912	0.409592820460911\\
1913	0.467693987626284\\
1914	0.380450436670171\\
1915	0.194299386629225\\
1916	0.259266058939441\\
1917	0.519984680916362\\
1918	0.604559131626093\\
1919	0.546050205094811\\
1920	0.6952774313474\\
1921	0.873111280853375\\
1922	0.801624105620298\\
1923	0.558907204646593\\
1924	0.217492902352898\\
1925	0.216626005739245\\
1926	0.536596589788816\\
1927	0.57157918963123\\
1928	0.545541628660225\\
1929	0.727889258677625\\
1930	0.836831375687247\\
1931	0.736799568615183\\
1932	0.591253411639962\\
1933	0.515553043125483\\
1934	0.468085569712334\\
1935	0.380608388740755\\
1936	0.239330342330169\\
1937	0.0993733391149632\\
1938	0.0709798726861005\\
1939	0.118839633150212\\
1940	0.169667368973162\\
1941	0.253631075213541\\
1942	0.291523198188598\\
1943	0.239451785393032\\
1944	0.172223080925841\\
1945	0.184309541688092\\
1946	0.255257386620725\\
1947	0.283402023958552\\
1948	0.316967899961962\\
1949	0.431514965653381\\
1950	0.564777164686913\\
1951	0.645180459793204\\
1952	0.637086985634417\\
1953	0.596256196278814\\
1954	0.605843435490969\\
1955	0.517635707686791\\
1956	0.294497922785423\\
1957	0.324262135264488\\
1958	0.408356308004844\\
1959	0.311590712120873\\
1960	0.393828413782572\\
1961	0.595912287640783\\
1962	0.668337842726345\\
1963	0.569063345909705\\
1964	0.384995559298892\\
1965	0.362748762668723\\
1966	0.376502307727034\\
1967	0.234134661314317\\
1968	0.126500134195505\\
1969	0.122826363133129\\
1970	0.111021296507554\\
1971	0.424401588387867\\
1972	0.527360455151492\\
1973	0.361004883662187\\
1974	0.212211727095242\\
1975	0.0993463450662114\\
1976	0.114767680587397\\
1977	0.18422711770695\\
1978	0.133071734823673\\
1979	0.500937504909985\\
1980	0.716611086135447\\
1981	0.603465594208894\\
1982	0.438592064201311\\
1983	0.385298656650456\\
1984	0.243723054779003\\
1985	0.244652261155884\\
1986	0.323480210615731\\
1987	0.212023710735374\\
1988	0.100990112849386\\
1989	0.460548149877939\\
1990	0.815147656390236\\
1991	1.05013948358504\\
1992	1.01448011845534\\
1993	0.682555541522697\\
1994	0.332111623332332\\
1995	0.338791186015819\\
1996	0.420806010476221\\
1997	0.476704336753123\\
1998	0.592357690576171\\
1999	0.704638587096014\\
2000	0.786202402541532\\
2001	0.756256052819903\\
2002	0.496562074509383\\
2003	0.131823377279541\\
2004	0.460403618072595\\
2005	0.66209804889265\\
2006	0.628445639809716\\
2007	0.609028936395521\\
2008	0.611478350901021\\
2009	0.509588377660989\\
2010	0.425546723387229\\
2011	0.377750026508615\\
2012	0.204281141944634\\
2013	0.101838359872192\\
2014	0.232136031336214\\
2015	0.182678272847518\\
2016	0.212508026160534\\
2017	0.331237474514543\\
2018	0.626238762958434\\
2019	0.862505803179862\\
2020	0.705570924169914\\
2021	0.234263582486387\\
2022	0.394989160463975\\
2023	0.576784963713645\\
2024	0.479135457555867\\
2025	0.304267367215268\\
2026	0.372912427980991\\
2027	0.752047804647977\\
2028	0.922444426393113\\
2029	0.685711955801617\\
2030	0.24562570012253\\
2031	0.0952149664980541\\
2032	0.29783468679509\\
2033	0.590454246447768\\
2034	0.713372745693316\\
2035	0.504037630331791\\
2036	0.232730949214758\\
2037	0.684905091770096\\
2038	0.983854994181434\\
2039	0.86612959554855\\
2040	0.552548626945925\\
2041	0.424085925989828\\
2042	0.342987607124723\\
2043	0.182277422292634\\
2044	0.263378332513842\\
2045	0.37339705784727\\
2046	0.39770390543536\\
2047	0.440903110517684\\
2048	0.432184697190136\\
2049	0.336929834621324\\
2050	0.153111949290317\\
2051	0.190154141735944\\
2052	0.488422625853265\\
2053	0.556308175341276\\
2054	0.454345965546173\\
2055	0.406025672573679\\
2056	0.277748980682957\\
2057	0.408016409257264\\
2058	0.672665700553558\\
2059	0.562657891573062\\
2060	0.165844927950249\\
2061	0.226230569495436\\
2062	0.381397103453215\\
2063	0.398357992112354\\
2064	0.348898053708721\\
2065	0.199209160623001\\
2066	0.149630487114522\\
2067	0.341238910873374\\
2068	0.439452639540974\\
2069	0.399455626740179\\
2070	0.364638146776064\\
2071	0.375433326516681\\
2072	0.316324732909768\\
2073	0.265462106513435\\
2074	0.40761072948684\\
2075	0.713412675452743\\
2076	0.872117135048252\\
2077	0.717198158472163\\
2078	0.573382077593219\\
2079	0.613824661400865\\
2080	0.463221315333943\\
2081	0.442760404335731\\
2082	0.656192158553778\\
2083	0.623269309932706\\
2084	0.314941893441745\\
2085	0.162276369665946\\
2086	0.347879821108773\\
2087	0.282177184246495\\
2088	0.151866910445077\\
2089	0.357461649860898\\
2090	0.696883964809186\\
2091	0.886512919331963\\
2092	0.752750695351503\\
2093	0.491751122268583\\
2094	0.375795575762789\\
2095	0.458583713601791\\
2096	0.626104700899833\\
2097	0.727104233572031\\
2098	0.700328622723721\\
2099	0.409465824542444\\
2100	0.342842033804764\\
2101	0.592318540671119\\
2102	0.41689207144557\\
2103	0.0948201377883429\\
2104	0.256871081727152\\
2105	0.0839211495914077\\
2106	0.236237257586215\\
2107	0.341970629691095\\
2108	0.183504208715006\\
2109	0.0615411121344424\\
2110	0.165289517183725\\
2111	0.136991944143538\\
2112	0.261850156241755\\
2113	0.439909953129805\\
2114	0.459837714317488\\
2115	0.280641864561617\\
2116	0.170410102926545\\
2117	0.405101635719437\\
2118	0.550822821523067\\
2119	0.510905048074835\\
2120	0.431140425310934\\
2121	0.469912582577482\\
2122	0.484404352670057\\
2123	0.397221213792665\\
2124	0.279638857507429\\
2125	0.293147832613153\\
2126	0.51318644236339\\
2127	0.789088411273616\\
2128	0.884472133177166\\
2129	0.757441280666994\\
2130	0.691697249170816\\
2131	0.769563388777948\\
2132	0.683504030759611\\
2133	0.476427582186031\\
2134	0.312775925542627\\
2135	0.322812468917375\\
2136	0.577117132394391\\
2137	0.671538354385498\\
2138	0.489016931595291\\
2139	0.315200645831363\\
2140	0.387282847015285\\
2141	0.386429320476764\\
2142	0.332136871545368\\
2143	0.241778395573182\\
2144	0.0755755275510679\\
2145	0.0473642939032983\\
2146	0.0980371660973691\\
2147	0.385651706018573\\
2148	0.504877666086254\\
2149	0.254855139333829\\
2150	0.253230266969263\\
2151	0.684287815295081\\
2152	0.822128912714925\\
2153	0.718821037164729\\
2154	0.466820253114424\\
2155	0.292583254554298\\
2156	0.521763955613788\\
2157	0.426447336667859\\
2158	0.103171173430472\\
2159	0.664606740922989\\
2160	0.762278560468122\\
2161	0.331364277219764\\
2162	0.462562328348095\\
2163	0.98835193895674\\
2164	1.11739484260639\\
2165	0.761358335848239\\
2166	0.157350133462438\\
2167	0.698617428207171\\
2168	0.93085849188666\\
2169	0.57458248944939\\
2170	0.062453830728176\\
2171	0.448974416161392\\
2172	0.365671542830458\\
2173	0.029594396929891\\
2174	0.138721512423533\\
2175	0.104263988273922\\
2176	0.363726463591261\\
2177	0.414214849521665\\
2178	0.218934352958625\\
2179	0.148655814597516\\
2180	0.307933234319595\\
2181	0.503958929032767\\
2182	0.621255468444863\\
2183	0.69683199149538\\
2184	0.732238881754616\\
2185	0.578049869989625\\
2186	0.319971612082135\\
2187	0.170393381136788\\
2188	0.252443376317474\\
2189	0.48514872189032\\
2190	0.657541219644766\\
2191	0.709313026065473\\
2192	0.670444089442109\\
2193	0.549566064174743\\
2194	0.354133796100334\\
2195	0.471601253955824\\
2196	0.707496768408721\\
2197	0.592365252078036\\
2198	0.205355538437627\\
2199	0.731703701783086\\
2200	1.11635931769367\\
2201	0.988475843326152\\
2202	0.478535623024466\\
2203	0.128111634252745\\
2204	0.297524454627772\\
2205	0.184052240951073\\
2206	0.050848051022857\\
2207	0.119540752401502\\
2208	0.137575648537866\\
2209	0.292470628245326\\
2210	0.395019720744952\\
2211	0.463078803358207\\
2212	0.650424902602407\\
2213	0.667772375160705\\
2214	0.326919052720565\\
2215	0.529017587620689\\
2216	0.767627821710124\\
2217	0.442416480470324\\
2218	0.310860369860339\\
2219	0.634400147979762\\
2220	0.431683285960708\\
2221	0.209885228398693\\
2222	0.282419784085626\\
2223	0.178053368746103\\
2224	0.298432114371034\\
2225	0.196947199536588\\
2226	0.607294369372543\\
2227	1.03948406290625\\
2228	1.04167748766783\\
2229	0.753231426440755\\
2230	0.402776324740174\\
2231	0.120285138695053\\
2232	0.234787859904334\\
2233	0.182372451872512\\
2234	0.356417302926723\\
2235	0.681481359156523\\
2236	0.695977308712279\\
2237	0.420887547490549\\
2238	0.0583259141512472\\
2239	0.313672772835691\\
2240	0.525302308890079\\
2241	0.484475268221283\\
2242	0.205212607084277\\
2243	0.102807867411926\\
2244	0.14986812951168\\
2245	0.0962270755835339\\
2246	0.307867811228832\\
2247	0.433291835866172\\
2248	0.501266829629842\\
2249	0.597903651802687\\
2250	0.622495874163556\\
2251	0.52283115192181\\
2252	0.439408825472825\\
2253	0.332961300681463\\
2254	0.170666368186098\\
2255	0.0934043663339444\\
2256	0.193689850270776\\
2257	0.531177759928281\\
2258	0.730642725062058\\
2259	0.661768121629161\\
2260	0.487934493874988\\
2261	0.323114665329564\\
2262	0.172680376478545\\
2263	0.365692864951645\\
2264	0.578719345323358\\
2265	0.665871986035861\\
2266	0.518791254583368\\
2267	0.163598494237292\\
2268	0.162437735697073\\
2269	0.225776200805259\\
2270	0.205101055111663\\
2271	0.241734602756543\\
2272	0.298523905106407\\
2273	0.454493222611064\\
2274	0.47043210528253\\
2275	0.3096070714294\\
2276	0.270696962518333\\
2277	0.262267719574812\\
2278	0.16317763018702\\
2279	0.117267226002107\\
2280	0.16944622336406\\
2281	0.378276923197142\\
2282	0.596510806442644\\
2283	0.722277600329865\\
2284	0.64104611149981\\
2285	0.319495949926115\\
2286	0.115836592216267\\
2287	0.253443882547611\\
2288	0.0807630124622878\\
2289	0.386102719568089\\
2290	0.651239411911451\\
2291	0.609181966840279\\
2292	0.444442502668448\\
2293	0.25604725884956\\
2294	0.194166067503929\\
2295	0.583484330338836\\
2296	0.665240307249094\\
2297	0.298240699109718\\
2298	0.385019270705891\\
2299	0.759038538743566\\
2300	0.681986405117584\\
2301	0.293060724931916\\
2302	0.159925783085217\\
2303	0.283380687229277\\
2304	0.51285349308703\\
2305	0.760497215174024\\
2306	0.689090841198422\\
2307	0.411640619072026\\
2308	0.332164061654763\\
2309	0.399723564452948\\
2310	0.467647959629338\\
2311	0.433314509287037\\
2312	0.39303268032868\\
2313	0.699265340997665\\
2314	0.988035104719019\\
2315	0.941426558618909\\
2316	0.573903445689738\\
2317	0.23496267406257\\
2318	0.426303262734387\\
2319	0.539376508845157\\
2320	0.458220010789885\\
2321	0.37302277998908\\
2322	0.416623011979994\\
2323	0.492234018049178\\
2324	0.54281279922741\\
2325	0.479898055718308\\
2326	0.404894211981506\\
2327	0.403139830149409\\
2328	0.326558591890118\\
2329	0.238406008424837\\
2330	0.0784024849615242\\
2331	0.337691253946841\\
2332	0.747027741188508\\
2333	0.848282226500438\\
2334	0.580955685934426\\
2335	0.214878329858553\\
2336	0.0226082810669926\\
2337	0.271824906894174\\
2338	0.658599259668353\\
2339	0.832200677930883\\
2340	0.616849507799825\\
2341	0.192161376680798\\
2342	0.380864237511746\\
2343	0.408768560969854\\
2344	0.121309477477796\\
2345	0.357233459892025\\
2346	0.575072275100789\\
2347	0.467072523045284\\
2348	0.2471292941859\\
2349	0.388191398798884\\
2350	0.38655299882563\\
2351	0.188557570834072\\
2352	0.095488776767105\\
2353	0.0564991837206079\\
2354	0.275113113645942\\
2355	0.385895940714511\\
2356	0.459548876941554\\
2357	0.589761680643968\\
2358	0.507052353648047\\
2359	0.197533454394597\\
2360	0.242865250306375\\
2361	0.226482174031624\\
2362	0.099390531389258\\
2363	0.440448939913211\\
2364	0.552545915978546\\
2365	0.432891550853693\\
2366	0.249394037601182\\
2367	0.0820959293159977\\
2368	0.130224369377083\\
2369	0.1998902260494\\
2370	0.313534276066218\\
2371	0.69554627705728\\
2372	0.970096804525628\\
2373	0.826650863009387\\
2374	0.292878291105766\\
2375	0.333951251561526\\
2376	0.647785063924078\\
2377	0.505560729828527\\
2378	0.260216901704878\\
2379	0.623250918537651\\
2380	0.687092492467244\\
2381	0.318560140750519\\
2382	0.336646622481062\\
2383	0.817582619470432\\
2384	1.09072654830709\\
2385	1.12131320123625\\
2386	0.966235550993818\\
2387	0.691671705193055\\
2388	0.348127636124951\\
2389	0.0524708160505785\\
2390	0.110249629840964\\
2391	0.0659114589994684\\
2392	0.184104759541748\\
2393	0.272912772276616\\
2394	0.31495306205772\\
2395	0.519520412159035\\
2396	0.604798019725406\\
2397	0.410123819455841\\
2398	0.126889487752279\\
2399	0.236473162106588\\
2400	0.366080652950078\\
2401	0.369696904915389\\
2402	0.291808411898205\\
2403	0.227644677970896\\
2404	0.192018538894076\\
2405	0.107115641809094\\
2406	0.0988545513996014\\
2407	0.211998645807942\\
2408	0.336127762397374\\
2409	0.664166188940645\\
2410	0.883372250873197\\
2411	0.630437089235707\\
2412	0.345332247794017\\
2413	0.902700239776331\\
2414	1.07923334420027\\
2415	0.680516922552722\\
2416	0.397870140882259\\
2417	0.791287820668918\\
2418	0.768139562661279\\
2419	0.371497036162821\\
2420	0.368875529067402\\
2421	0.56499292309332\\
2422	0.629640932911192\\
2423	0.734424962463665\\
2424	0.739393275269529\\
2425	0.537304288935762\\
2426	0.316172260500193\\
2427	0.315672260381073\\
2428	0.375219228712145\\
2429	0.502920647514665\\
2430	0.605308144819928\\
2431	0.583639437847615\\
2432	0.414909660969024\\
2433	0.121749759775156\\
2434	0.201878352447949\\
2435	0.373096767206004\\
2436	0.383172105725202\\
2437	0.387475110686276\\
2438	0.417177181295638\\
2439	0.481453471950417\\
2440	0.561015021098289\\
2441	0.517791057592086\\
2442	0.411551782655617\\
2443	0.423151798210766\\
2444	0.493856870730523\\
2445	0.557083804698006\\
2446	0.665308490078401\\
2447	0.816002676905285\\
2448	0.865953192593414\\
2449	0.743418563750632\\
2450	0.568171574352606\\
2451	0.394375782862998\\
2452	0.170008500504822\\
2453	0.306213135265175\\
2454	0.432293399316723\\
2455	0.403401712739727\\
2456	0.334151893760392\\
2457	0.194316755662899\\
2458	0.113552200323938\\
2459	0.333603361534355\\
2460	0.337782567218056\\
2461	0.38425432943479\\
2462	0.697508176251924\\
2463	0.878463655474624\\
2464	0.849348559096258\\
2465	0.761827863660482\\
2466	0.770222872610293\\
2467	0.805673965052126\\
2468	0.720479639718878\\
2469	0.516797210521448\\
2470	0.309850046942156\\
2471	0.270494065658949\\
2472	0.248406187918424\\
2473	0.122816790824152\\
2474	0.238803042916012\\
2475	0.244373768213053\\
2476	0.122071714380395\\
2477	0.395988814970336\\
2478	0.507910095918447\\
2479	0.431889803422749\\
2480	0.560495236119031\\
2481	0.724319958506055\\
2482	0.548931372843372\\
2483	0.152128399029537\\
2484	0.351097112765379\\
2485	0.452487010879978\\
2486	0.388721221600152\\
2487	0.604879483693088\\
2488	0.684675015679223\\
2489	0.357252809191404\\
2490	0.218984426214483\\
2491	0.510290667970156\\
2492	0.419818487541163\\
2493	0.125697533273625\\
2494	0.422545608806378\\
2495	0.561544498752838\\
2496	0.3446779248682\\
2497	0.169792397022498\\
2498	0.51288113363929\\
2499	0.418709333812445\\
2500	0.192834917418304\\
2501	0.62259458653702\\
2502	0.63440902345212\\
2503	0.250468722600444\\
2504	0.187840317611922\\
2505	0.171628203306892\\
2506	0.318935787075608\\
2507	0.562777927095789\\
2508	0.648690216664694\\
2509	0.670618521339283\\
2510	0.683241273363139\\
2511	0.666353589379494\\
2512	0.521988992218709\\
2513	0.160816736405685\\
2514	0.351924237207107\\
2515	0.673287218568801\\
2516	0.712987123386769\\
2517	0.538410848144212\\
2518	0.298837660328634\\
2519	0.191844228308135\\
2520	0.311385160359598\\
2521	0.485842555687534\\
2522	0.547861872405787\\
2523	0.435939607253063\\
2524	0.290739171773254\\
2525	0.214004953371685\\
2526	0.135160136873069\\
2527	0.180700922755539\\
2528	0.211678761107281\\
2529	0.340017328705734\\
2530	0.828272892174902\\
2531	1.12221076460034\\
2532	0.894377270243091\\
2533	0.451105766864189\\
2534	0.683652833269965\\
2535	0.755543343208853\\
2536	0.360592077547043\\
2537	0.312399270234679\\
2538	0.663811346931144\\
2539	0.683060385789572\\
2540	0.476102207851042\\
2541	0.344195385279901\\
2542	0.485906783068071\\
2543	0.599507674302956\\
2544	0.47829499747364\\
2545	0.378868379888887\\
2546	0.55599961753804\\
2547	0.574884826277062\\
2548	0.355305398697653\\
2549	0.0553431736529224\\
2550	0.183520029121514\\
2551	0.170885729618703\\
2552	0.204727907781408\\
2553	0.521280915555281\\
2554	0.659360199933752\\
2555	0.474241580341751\\
2556	0.409935260936763\\
2557	0.57670501565337\\
2558	0.492060628646494\\
2559	0.36104649753203\\
2560	0.467277055127596\\
2561	0.53568810861621\\
2562	0.532188077797692\\
2563	0.611887873453718\\
2564	0.751840343520059\\
2565	0.782423615907806\\
2566	0.787886150030728\\
2567	0.798875618818168\\
2568	0.599547186751193\\
2569	0.346547529034041\\
2570	0.392833588330027\\
2571	0.371545838516638\\
2572	0.33042668012592\\
2573	0.376482151247394\\
2574	0.326254500670693\\
2575	0.154553082577117\\
2576	0.175409790085204\\
2577	0.414374557476191\\
2578	0.425169635144266\\
2579	0.2384548901923\\
2580	0.302883836508541\\
2581	0.258062855905528\\
2582	0.0688460134669603\\
2583	0.376914888830578\\
2584	0.407222257958267\\
2585	0.185179204884616\\
2586	0.125907688987891\\
2587	0.162074141855665\\
2588	0.445835614663616\\
2589	0.733252500862176\\
2590	0.757253801816232\\
2591	0.611052762506087\\
2592	0.49569111812248\\
2593	0.502527698115421\\
2594	0.682575543766505\\
2595	0.720503370922768\\
2596	0.594911914495004\\
2597	0.735336495379729\\
2598	0.934988819573801\\
2599	0.858645894631095\\
2600	0.545666403168359\\
2601	0.106041426812709\\
2602	0.385226353419871\\
2603	0.775567690556598\\
2604	0.953013227341693\\
2605	0.797435489488455\\
2606	0.444127674679184\\
2607	0.400441362606865\\
2608	0.451004799828556\\
2609	0.553960093038017\\
2610	0.806973401673537\\
2611	0.741596596123361\\
2612	0.274666966383233\\
2613	0.296717036776252\\
2614	0.619002580275588\\
2615	0.644234691579102\\
2616	0.480110713106261\\
2617	0.246377955919503\\
2618	0.0683525752982615\\
2619	0.368472476455402\\
2620	0.46098879869468\\
2621	0.2777847462838\\
2622	0.28366808210918\\
2623	0.453078340245773\\
2624	0.433732234178137\\
2625	0.506655445350639\\
2626	0.526407726712679\\
2627	0.305646286961482\\
2628	0.0381902323852245\\
2629	0.100498103776612\\
2630	0.0522309420787739\\
2631	0.314033731016869\\
2632	0.530637316799885\\
2633	0.531304658787448\\
2634	0.293474850418173\\
2635	0.0437967243774548\\
2636	0.0685759149619274\\
2637	0.373398188974961\\
2638	0.635981779650688\\
2639	0.530152456764023\\
2640	0.326806121608259\\
2641	0.705457624000179\\
2642	0.840525757832819\\
2643	0.630881149796973\\
2644	0.58076969156397\\
2645	0.60646726569733\\
2646	0.423041501247986\\
2647	0.461032931389744\\
2648	0.512333711495498\\
2649	0.385158808101022\\
2650	0.156385575279112\\
2651	0.239257315727519\\
2652	0.662085314942432\\
2653	0.74848356233487\\
2654	0.319258252290664\\
2655	0.385127187977205\\
2656	0.86105524113669\\
2657	0.980533731981672\\
2658	0.900802832255687\\
2659	0.819684329772105\\
2660	0.724804645637223\\
2661	0.57842243864649\\
2662	0.487438767004113\\
2663	0.438075223419736\\
2664	0.179660127849226\\
2665	0.2752434331927\\
2666	0.594248198399598\\
2667	0.650031510435683\\
2668	0.497914032149186\\
2669	0.396569042746096\\
2670	0.418333683778863\\
2671	0.526732038113499\\
2672	0.687449461652653\\
2673	0.683014080953689\\
2674	0.486508465574287\\
2675	0.261629687080514\\
2676	0.190393360874163\\
2677	0.275272573765048\\
2678	0.387376056348064\\
2679	0.379916498305273\\
2680	0.238474995380166\\
2681	0.0976448626022452\\
2682	0.173500846942502\\
2683	0.226983018980136\\
2684	0.248886002387093\\
2685	0.289954499995025\\
2686	0.309002442785036\\
2687	0.315943498022671\\
2688	0.320539887078342\\
2689	0.241502686067227\\
2690	0.340151956017924\\
2691	0.696528142061882\\
2692	0.837023465677118\\
2693	0.612157648175777\\
2694	0.332106199611008\\
2695	0.492919623866374\\
2696	0.60479366357409\\
2697	0.430334793240443\\
2698	0.277260353369746\\
2699	0.633460094580185\\
2700	0.815582688669703\\
2701	0.62214547845975\\
2702	0.17339349821434\\
2703	0.279116174127969\\
2704	0.534642528849369\\
2705	0.621473547225715\\
2706	0.564404717471551\\
2707	0.456130723852707\\
2708	0.353425692766235\\
2709	0.36221043676556\\
2710	0.579116836164626\\
2711	0.597465703042163\\
2712	0.253296859672558\\
2713	0.422839542069578\\
2714	0.779012982230475\\
2715	0.708066642192159\\
2716	0.368354797378901\\
2717	0.156296734473523\\
2718	0.149633141157318\\
2719	0.570213311503524\\
2720	0.922925798761838\\
2721	0.812529263806461\\
2722	0.252681226225587\\
2723	0.3573004218989\\
2724	0.530479218777974\\
2725	0.253041589100612\\
2726	0.516262001330654\\
2727	0.797021906416936\\
2728	0.708978937395754\\
2729	0.477894711355259\\
2730	0.430230524913688\\
2731	0.451328961593164\\
2732	0.418030494766025\\
2733	0.397779965782937\\
2734	0.401475445465863\\
2735	0.363674414385087\\
2736	0.345976191143251\\
2737	0.326896893448495\\
2738	0.254113995515989\\
2739	0.250438842241948\\
2740	0.346340611807522\\
2741	0.356068433323898\\
2742	0.310840742878802\\
2743	0.332250226177788\\
2744	0.292090963911894\\
2745	0.258675086533339\\
2746	0.280779822406104\\
2747	0.298118975566392\\
2748	0.336631395057401\\
2749	0.360594010847083\\
2750	0.314414546506857\\
2751	0.21660852958448\\
2752	0.38384868598016\\
2753	0.655727246885472\\
2754	0.715238638585038\\
2755	0.55545028378814\\
2756	0.372169904487277\\
2757	0.354248902673036\\
2758	0.441846801230407\\
2759	0.537519238010552\\
2760	0.580982054397361\\
2761	0.522336454592111\\
2762	0.451342741626802\\
2763	0.507707328826896\\
2764	0.428981167463387\\
2765	0.215552196750038\\
2766	0.348106667365461\\
2767	0.540682961389417\\
2768	0.587304202197182\\
2769	0.530464545603214\\
2770	0.560728457355743\\
2771	0.602409824112075\\
2772	0.498721648394185\\
2773	0.33480424181748\\
2774	0.474343206933372\\
2775	0.699301483781716\\
2776	0.746404306760598\\
2777	0.633391180575406\\
2778	0.42151326799198\\
2779	0.156089782397325\\
2780	0.33307525499435\\
2781	0.600743982349584\\
2782	0.65896697367475\\
2783	0.506798631429925\\
2784	0.256895178778491\\
2785	0.116296024664021\\
2786	0.12970219095603\\
2787	0.311769105953759\\
2788	0.483080353030837\\
2789	0.444101023455624\\
2790	0.187403371531068\\
2791	0.240663381783499\\
2792	0.448927946780187\\
2793	0.444261205113396\\
2794	0.290179015425789\\
2795	0.181586839897948\\
2796	0.26284935677442\\
2797	0.239214104451046\\
2798	0.0696289245102543\\
2799	0.131157546962834\\
2800	0.229771413898352\\
2801	0.670330927555257\\
2802	1.19256187669762\\
2803	1.28264138678125\\
2804	0.933810642800727\\
2805	0.751122698267988\\
2806	0.648615863363029\\
2807	0.264795715538412\\
2808	0.52006294875539\\
2809	0.611919630362198\\
2810	0.260453412169309\\
2811	0.344012311398092\\
2812	0.492265440915722\\
2813	0.4768219740966\\
2814	0.469383205521252\\
2815	0.305531686862305\\
2816	0.0524515221576862\\
2817	0.297505876113823\\
2818	0.241807516256212\\
2819	0.0980210517303245\\
2820	0.387689126467692\\
2821	0.325218776167045\\
2822	0.0894888123822346\\
2823	0.408765089214237\\
2824	0.526383202701153\\
2825	0.581676335996294\\
2826	0.558341409989072\\
2827	0.282178026090038\\
2828	0.281688181438658\\
2829	0.456137799845025\\
2830	0.273114186479839\\
2831	0.0761851601868473\\
2832	0.24949233335317\\
2833	0.12976420086205\\
2834	0.155947958090168\\
2835	0.208055682730807\\
2836	0.116272768930604\\
2837	0.292909161168547\\
2838	0.265747601699596\\
2839	0.359351360565508\\
2840	0.591766718738769\\
2841	0.452800145237731\\
2842	0.22905521124639\\
2843	0.466426496073556\\
2844	0.359610427024282\\
2845	0.423663141231086\\
2846	0.864208439994591\\
2847	0.957587662970971\\
2848	0.657114749571923\\
2849	0.315075249287783\\
2850	0.297898827335598\\
2851	0.169796107472986\\
2852	0.0965456411912914\\
2853	0.240610978693463\\
2854	0.166943183894961\\
2855	0.066275208442539\\
2856	0.178257905147472\\
2857	0.20815071723224\\
2858	0.249412806966923\\
2859	0.296211451763708\\
2860	0.277507252207846\\
2861	0.108326387131583\\
2862	0.310264508077565\\
2863	0.59413611618801\\
2864	0.58557806747698\\
2865	0.307604588754809\\
2866	0.22115680502879\\
2867	0.432563430702944\\
2868	0.556505291105714\\
2869	0.792226863395745\\
2870	1.01461055235772\\
2871	0.954135388551596\\
2872	0.5566830698251\\
2873	0.114411546590803\\
2874	0.366508523270216\\
2875	0.496298070276572\\
2876	0.437444349213079\\
2877	0.262510228472639\\
2878	0.203040195929731\\
2879	0.320668989298008\\
2880	0.264109939674988\\
2881	0.195004330463072\\
2882	0.571080236427599\\
2883	0.753777541539555\\
2884	0.589007908795908\\
2885	0.334953501116691\\
2886	0.297129898061457\\
2887	0.206953093931149\\
2888	0.267501735858025\\
2889	0.457763044937252\\
2890	0.496817598029632\\
2891	0.418161353902106\\
2892	0.298714618249832\\
2893	0.19163755082604\\
2894	0.223882665285297\\
2895	0.266247989674296\\
2896	0.367576624098085\\
2897	0.63932156278976\\
2898	0.84993018376426\\
2899	0.859536093942835\\
2900	0.667788771677902\\
2901	0.396304733913131\\
2902	0.196831652500682\\
2903	0.0550782970937651\\
2904	0.291099363330379\\
2905	0.689733636638862\\
2906	0.921323659690369\\
2907	0.757424478566389\\
2908	0.3387211918822\\
2909	0.198872237340901\\
2910	0.177587564974206\\
2911	0.212409044581962\\
2912	0.340899024066134\\
2913	0.227924840134843\\
2914	0.373505106174537\\
2915	0.619180990811718\\
2916	0.496286922307965\\
2917	0.18274063043134\\
2918	0.325132982200334\\
2919	0.341292579609184\\
2920	0.129470304118648\\
2921	0.267093876392548\\
2922	0.421754342855786\\
2923	0.340396052587629\\
2924	0.129866907112207\\
2925	0.350607042469633\\
2926	0.694287975583796\\
2927	1.00175861006277\\
2928	1.10223701606333\\
2929	0.861516791889041\\
2930	0.381574001963019\\
2931	0.384732924232797\\
2932	0.617214347287528\\
2933	0.569778395128849\\
2934	0.52711614462541\\
2935	0.678188105946963\\
2936	0.660106318351845\\
2937	0.376841331971459\\
2938	0.10427871540024\\
2939	0.0733825647106722\\
2940	0.175035459973223\\
2941	0.19401079734211\\
2942	0.355146900964947\\
2943	0.712820226608551\\
2944	0.838174899778083\\
2945	0.619494047498235\\
2946	0.303120239397868\\
2947	0.149308681776949\\
2948	0.0243991047679793\\
2949	0.180085499759422\\
2950	0.271498214584984\\
2951	0.313853247172306\\
2952	0.319548437122617\\
2953	0.323558680469762\\
2954	0.344492399183112\\
2955	0.270503071164872\\
2956	0.401149398292405\\
2957	0.557023540019568\\
2958	0.551389418748485\\
2959	0.476849146236668\\
2960	0.286010126319955\\
2961	0.102930097044064\\
2962	0.324571319751667\\
2963	0.192899548226134\\
2964	0.36287745150451\\
2965	0.805807317645421\\
2966	0.873295132833967\\
2967	0.55723363928397\\
2968	0.202336414822764\\
2969	0.387464302566498\\
2970	0.537101466626018\\
2971	0.518753232661152\\
2972	0.473549436934839\\
2973	0.332089270094607\\
2974	0.0620998640567947\\
2975	0.444774499452775\\
2976	0.659008969713915\\
2977	0.56426400051541\\
2978	0.259146086125496\\
2979	0.142677608087048\\
2980	0.39801406819677\\
2981	0.614835418795059\\
2982	0.700859296157391\\
2983	0.498076524037185\\
2984	0.213498315489178\\
2985	0.463609109651608\\
2986	0.472884944454715\\
2987	0.275157898799295\\
2988	0.312890885916419\\
2989	0.331437125710095\\
2990	0.308685770263008\\
2991	0.329184756365327\\
2992	0.268861801964936\\
2993	0.320307230331213\\
2994	0.415991934497969\\
2995	0.363766916706111\\
2996	0.249452588671153\\
2997	0.290640171135724\\
2998	0.259815978450723\\
2999	0.240276182124714\\
3000	0.50255178964098\\
3001	0.508313899260333\\
3002	0.164718508571872\\
3003	0.522915590160503\\
3004	0.771752522248988\\
3005	0.56254376888211\\
3006	0.414375044181577\\
3007	0.58530936428231\\
3008	0.525934775603493\\
3009	0.517571607829532\\
3010	0.584267414300266\\
3011	0.423843963608353\\
3012	0.214031241076034\\
3013	0.210909721066265\\
3014	0.10039017259611\\
3015	0.114559134585482\\
3016	0.293564006433962\\
3017	0.387689754176915\\
3018	0.415893236958748\\
3019	0.364075658135544\\
3020	0.188468549569863\\
3021	0.263997097986603\\
3022	0.364662535765155\\
3023	0.178353129179622\\
3024	0.161894283137004\\
3025	0.356891832104402\\
3026	0.295359235681265\\
3027	0.186123543332393\\
3028	0.190235082999813\\
3029	0.111074955555478\\
3030	0.181061604237197\\
3031	0.290839329203146\\
3032	0.29209563681503\\
3033	0.172197136019273\\
3034	0.0701765228018253\\
3035	0.174331032138581\\
3036	0.193987705340803\\
3037	0.261793267648201\\
3038	0.298778464888592\\
3039	0.207972519068955\\
3040	0.224420026212278\\
3041	0.292880680914289\\
3042	0.284644136052941\\
3043	0.272414235829225\\
3044	0.172093235982088\\
3045	0.211518872086928\\
3046	0.2398352474982\\
3047	0.0570453323149805\\
3048	0.145282404323055\\
3049	0.143314229452618\\
3050	0.0713122194523107\\
3051	0.0939078707898774\\
3052	0.046759848563673\\
3053	0.213594543908243\\
3054	0.252189321696122\\
3055	0.151646905920037\\
3056	0.159099973049887\\
3057	0.198092297794487\\
3058	0.15071311848097\\
3059	0.0647484669779673\\
3060	0.0713274503295508\\
3061	0.137765791177349\\
3062	0.0986943169385271\\
3063	0.0445669715821388\\
3064	0.0631381521904067\\
3065	0.019282669215936\\
3066	0.101984618885181\\
3067	0.140609074396725\\
3068	0.123823841199595\\
3069	0.136033038507269\\
3070	0.140307233154853\\
3071	0.096087459489906\\
3072	0.112631658587905\\
3073	0.126818978206194\\
3074	0.0301694064531703\\
3075	0.0941102422948428\\
3076	0.122705156700126\\
3077	0.09373565132262\\
3078	0.0842835712084249\\
3079	0.0539243133658129\\
};
\addlegendentry{abs(xcorr(.,.))}

\end{axis}
\end{tikzpicture}%
	\caption{Preamble Correlation in WARP Experiment}
	\label{fig:warp_preamble_corr}
\end{figure}
