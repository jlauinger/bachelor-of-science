%% -----------------------------------------------------------------------------

\chapter{Conclusion}\label{ch:Conclusion}
\glsresetall % Resets all acronyms to not used

Recognizing senders by correlating their MAC addresses in an IEEE 802.11 a/g frame collision is an approach that to my knowledge has not been explored before. In this thesis, I proposed an algorithm to apply this technique and developed a proof-of-concept implementation using Matlab with the Communications and WLAN Systems Toolboxes.\\

The technique promiscuously listens on the wireless network interface to cache MAC addresses of stations connected to the network. For each of these addresses, several reference signals are pre-modulated and stored, considering different \glspl{MCS} and scrambler initialization values. When a collision is received, samples containing the MAC address are cross-correlated in the time-domain. The reference signals with highest correlations are used to determine the senders participating in the collided transmission.\\

I evaluated the performance and accuracy of this algorithm using simulations. Even for higher \glspl{MCS}, the technique performed very well. Varying the scrambler initialization caused severe regression, whereas the destination MAC address, preceding the sender, played a minor role. Simulations with standard channel models showed measurable impact, however up to a certain point sender detection retained a reasonable accuracy.\\

Experiments with real hardware, \gls{WARP} \glspl{SDR} in particular, unveiled serious problems with the proposed algorithm. While receiving a collision, correlating the frame preambles, and measuring a delay between two collided frames worked well and as expected, detecting sender MAC addresses did not function at all.\\

The proposed algorithm is not directly applicable to current versions of the IEEE 802.11 standard using \gls{MIMO}. This is due to possible spatial fractioning of the time-domain samples containing the sender MAC address, rendering naive cross-correlation useless.\\

Future work could quantise channel effects on real hardware, and adapt the detection technique to modern IEEE 802.11 n/ac/ax standards and \gls{MIMO} transmissions.

In summary, sender detection by using cross-correlation in the time-domain worked reasonably well despite some problems and should be explored in more detail.
