% **************************************************************************************************************
% A Classic Thesis Style
% An Homage to The Elements of Typographic Style
%
% Copyright (C) 2015 André Miede http://www.miede.de
%
% If you like the style then I would appreciate a postcard. My address 
% can be found in the file ClassicThesis.pdf. A collection of the 
% postcards I received so far is available online at 
% http://postcards.miede.de
%
% License:
% This program is free software; you can redistribute it and/or modify
% it under the terms of the GNU General Public License as published by
% the Free Software Foundation; either version 2 of the License, or
% (at your option) any later version.
%
% This program is distributed in the hope that it will be useful,
% but WITHOUT ANY WARRANTY; without even the implied warranty of
% MERCHANTABILITY or FITNESS FOR A PARTICULAR PURPOSE.  See the
% GNU General Public License for more details.
%
% You should have received a copy of the GNU General Public License
% along with this program; see the file COPYING.  If not, write to
% the Free Software Foundation, Inc., 59 Temple Place - Suite 330,
% Boston, MA 02111-1307, USA.
%
% **************************************************************************************************************
\RequirePackage{fix-cm} % fix some latex issues see: http://texdoc.net/texmf-dist/doc/latex/base/fixltx2e.pdf
\documentclass[ twoside,openright,titlepage,numbers=noenddot,headinclude,%1headlines,% letterpaper a4paper
                footinclude=true,cleardoublepage=empty,abstractoff, % <--- obsolete, remove (todo)
                BCOR=5mm,paper=a4,fontsize=11pt,%11pt,a4paper,%
                ngerman,american,%
                ]{scrreprt}

% ****************************************************************************************************
% Set the encoding of your files. UTF-8 is the only sensible encoding nowadays. If you can't read
% äöüßáéçèê∂åëæƒÏ€ then change the encoding setting in your editor, not the line below. If your editor
% does not support utf8 use another editor!
% ****************************************************************************************************
\PassOptionsToPackage{utf8}{inputenc}
	\usepackage{inputenc}
	
\usepackage{etoolbox}

% ****************************************************************************************************
% Insert the information about your thesis here
% ****************************************************************************************************
\newcommand{\myTitle}{Title\xspace}
\newcommand{\myDegree}{Master Thesis\xspace}
\newcommand{\myName}{Name of the Student\xspace}
\newcommand{\myProf}{Prof. Dr.-Ing. Matthias Hollick\xspace}
%\newcommand{\myOtherProf}{Second advisor\xspace}
\newcommand{\mySupervisor}{Marc Werner and Matthias Schulz\xspace}
\newcommand{\myThesiscode}{SEEMOO-MSC-$0000$\xspace} % You will get this from your supervisor
\newcommand{\myFaculty}{Department of Computer Science\xspace}
\newcommand{\myDepartment}{Secure Mobile Networking Lab\xspace}
\newcommand{\myUni}{\protect{Technische Universität Darmstadt}\xspace}
\newcommand{\myLocation}{Darmstadt\xspace}
\newcommand{\myTime}{\formatdate{24}{04}{1337}\xspace}
\newcommand{\myVersion}{0.1\xspace}

% Choose if you want the standard template with enough space for margin notes or the tempate with small margins
\newtoggle{adrianstyle}
%\toggletrue{adrianstyle} % uncomment this line to have smaller margins
\togglefalse{adrianstyle} % uncomment this line for the standard seemoo template

%********************************************************************
% Note: Make all your adjustments in here
%*******************************************************
\input{classicthesis-config}
\usepackage{tikz}
\usetikzlibrary{dsp,chains}
\usepackage{pgfplots}
%\pgfplotsset{compat=newest}
\usetikzlibrary{positioning}
\DeclareMathAlphabet{\mathpzc}{OT1}{pzc}{m}{it}
\newcommand{\z}{\mathpzc{z}}
\usepackage{float}
\usepackage{scalefnt}

\usepackage{lipsum}

% To cache tikz pictures you have to run pdflatex with -shell-escape or --enable-write18
\ifnum\pdfshellescape=1
\usepgfplotslibrary{external}
\tikzexternalize[prefix=gfxcompiled/]
%\tikzset{external/remake next}
%\tikzset{external/force remake}
\newcommand{\tikzremakenext}{\tikzset{external/remake next}}
%\tikzexternalize[shell escape=-enable-write18]
\else
\newcommand{\tikzremakenext}{}
\fi

%lengths for matlab2tikz
\newlength\figureheight
\newlength\figurewidth


\usepackage{textpos}
\usepackage{datetime}
\usepackage{changepage}

% move lstlisting captions to the left
%\usepackage{caption}
%\captionsetup[lstlisting]{singlelinecheck=off}

% setup Matlab syntax highlighting
\lstset{language=Matlab,%
    breaklines=true,%
    morekeywords={matlab2tikz},
    keywordstyle=\color{blue},%
    morekeywords=[2]{1}, keywordstyle=[2]{\color{black}},
    identifierstyle=\color{black},%
    stringstyle=\color{mylilas},
    commentstyle=\color{mygreen},%
    showstringspaces=false,%without this there will be a symbol in the places where there is a space
    numbers=left,%
    numberstyle={\tiny \color{black}},% size of the numbers
    numbersep=9pt, % this defines how far the numbers are from the text
    emph=[1]{for,end,break},emphstyle=[1]\color{red}, %some words to emphasise
    %emph=[2]{word1,word2}, emphstyle=[2]{style},
}


%********************************************************************
% Bibliographies
%*******************************************************
\addbibresource{Bibliography.bib}

%********************************************************************
% Hyphenation
%*******************************************************
%\hyphenation{put special hyphenation here}
\hyphenation{
	Si-mu-link
	OO-WARP-Lab
	WARP-Lab
}

% ********************************************************************
% GO!GO!GO! MOVE IT!
%*******************************************************
\begin{document}
\frenchspacing
\raggedbottom
\selectlanguage{american} % american ngerman
%\renewcommand*{\bibname}{new name}
%\setbibpreamble{}
\pagenumbering{roman}
\pagestyle{plain}
%********************************************************************
% Frontmatter
%*******************************************************
%\include{FrontBackmatter/DirtyTitlepage}
%*******************************************************
% Titlepage
%*******************************************************
\begin{titlepage}

    %\begin{comment}
    %\begin{textblock*}{297mm}(0mm,0mm)
    %    \includegraphics[width=\paperwidth]{gfx/titlePage}
    %\end{textblock*}
    %\phantom{Invisible, but important}
    %\newpage
    %\end{comment}

	% if you want the titlepage to be centered, uncomment and fine-tune the line below (KOMA classes environment)
	\begin{addmargin}[-0.5cm]{\iftoggle{adrianstyle}{-2cm}{-3.5cm}}
    \begin{center}
        \large

        \includegraphics[width=6cm]{gfx/logos/tud_logo}
        
        \vfill

        \begingroup
            %\color{Maroon}\spacedallcaps{\myTitle} \\ \bigskip
            \color{Maroon}\spacedallcaps{\myTitle} \bigskip
        \endgroup

        \spacedlowsmallcaps{\myName}

        \vfill

        \medskip

        \myDegree \\ \medskip
        \myTime

        \bigskip

        \vfill

        \myDepartment \\
        \myFaculty \\[0.2cm]
        %\myUni \\
        \includegraphics[width=5cm]{gfx/logos/seemoo_logo} \\

        %\vfill

    \end{center}
    \end{addmargin}
\end{titlepage}   
\thispagestyle{empty}
\begin{adjustwidth}{\iftoggle{adrianstyle}{-1.75cm}{-4cm}}{}
\noindent\myTitle \\
\noindent\myDegree \\
\noindent\myThesiscode

\bigskip

\noindent Submitted by \myName \\
\noindent Date of submission: \myTime

\bigskip

\noindent Advisor: \myProf

\noindent Supervisor: \mySupervisor


\hfill

\vfill

\noindent \myUni \\
\noindent \myFaculty \\
\noindent \myDepartment \\
\end{adjustwidth}
%\cleardoublepage\include{FrontBackmatter/Dedication}
%\cleardoublepage\include{FrontBackmatter/Foreword}
\cleardoublepage%*******************************************************
% Abstract
%*******************************************************
%\renewcommand{\abstractname}{Abstract}
\pdfbookmark[1]{Abstract}{Abstract}
\begingroup
\let\clearpage\relax
\let\cleardoublepage\relax
\let\cleardoublepage\relax

\chapter*{Abstract}
\lipsum[1]

\vfill

\selectlanguage{ngerman}
\pdfbookmark[1]{Zusammenfassung}{Zusammenfassung}
\chapter*{Zusammenfassung}
\lipsum[2]

\selectlanguage{american}

\endgroup			

\vfill
%\cleardoublepage\include{FrontBackmatter/Publication}
%\cleardoublepage%*******************************************************
% Acknowledgments
%*******************************************************
\pdfbookmark[1]{Acknowledgments}{acknowledgments}

%\begin{flushright}{\slshape    
%    We have seen that computer programming is an art, \\ 
%    because it applies accumulated knowledge to the world, \\ 
%    because it requires skill and ingenuity, and especially \\
%    because it produces objects of beauty.} \\ \medskip
%    --- \defcitealias{knuth:1974}{Donald E. Knuth}%\citetalias{knuth:1974} \citep{knuth:1974}
%\end{flushright}



\bigskip
\ \\[5cm]
\begingroup
\let\clearpage\relax
\let\cleardoublepage\relax
\let\cleardoublepage\relax
\centering
\begin{minipage}[h]{7cm}
\chapter*{Acknowledgments}
%\ \\[5cm]
\centering
%\begin{minipage}[h]{7cm}
{\slshape 
I would like to express my deepest gratitude to my parents and my family for supporting me in all the years of my studies and also while writing this thesis.

\bigskip

Special thanks for giving helpful advice while writing this thesis goes to Prof. Matthias Hollick and Adrian Loch.

\bigskip

Furthermore, I especially thank Sandrine Adéla\"ide and Adrian Loch for proofreading my thesis.
}
\end{minipage}

\endgroup



\pagestyle{scrheadings}
\cleardoublepage\include{FrontBackmatter/Contents}
%********************************************************************
% Mainmatter
%*******************************************************
\pagenumbering{arabic}
%\setcounter{page}{90}
% use \cleardoublepage here to avoid problems with pdfbookmark
\cleardoublepage

\ctparttext{The first chapter of this part gives an introduction and a motivation to this thesis, followed by a presentation of related work found in the area of physical layer security. In the third chapter, we present some definitions and background information to make it easier for the reader to quickly understand the subsequent parts of this thesis.}
\part{Introduction}
%************************************************
\chapter{Introduction}\label{ch:introduction}
%************************************************
\glsresetall % Resets all acronyms to not used

Start a chapter with text and not with a section header. Open the
\emph{classicthesis-config.tex} file to insert the title of your thesis, the
names of your supervisors and the hand-in date of your thesis.

\section{First Section}
\label{sec:first_section}

After a section there should always be text before the next section. The first
paragraph is always without indentation. Starting from the second paragraph,
there is an indentation.

Here is an equation without numbers for referencing:
\begin{align*}
\underbrace{\begin{pmatrix}\mathcal{B}_1\\\mathcal{B}_2\\\vdots\\\mathcal{B}_R\end{pmatrix}}_\mathcal{B} &= \underbrace{\begin{pmatrix}H_{1,1} & H_{1,2} & \hdots & H_{1,T}\\H_{2,1} & H_{2,2} & \hdots & H_{2,T}\\\vdots & \vdots & \ddots & \vdots\\H_{R,1} & H_{R,2} & \hdots & H_{R,T}\end{pmatrix}}_{H_{A\rightarrow B}}\cdot \underbrace{\begin{pmatrix}\mathcal{A}_1\\\mathcal{A}_2\\\vdots\\\mathcal{A}_T\end{pmatrix}}_\mathcal{A}
\end{align*}

Here is an equation that you can reference:
\begin{align}
\underbrace{\begin{pmatrix}\mathcal{B}_1\\\mathcal{B}_2\\\vdots\\\mathcal{B}_R\end{pmatrix}}_\mathcal{B} &= \underbrace{\begin{pmatrix}H_{1,1} & H_{1,2} & \hdots & H_{1,T}\\H_{2,1} & H_{2,2} & \hdots & H_{2,T}\\\vdots & \vdots & \ddots & \vdots\\H_{R,1} & H_{R,2} & \hdots & H_{R,T}\end{pmatrix}}_{H_{A\rightarrow B}}\cdot \underbrace{\begin{pmatrix}\mathcal{A}_1\\\mathcal{A}_2\\\vdots\\\mathcal{A}_T\end{pmatrix}}_\mathcal{A}\label{eqn:example}
\end{align}

\subsection{Referencing}

Take a look in the following list to reference sections, figures and equations:
\begin{itemize}
  \item \autoref{sec:first_section}
  \item \autoref{fig:wiretapchannel}
  \item \autoref{eqn:example}
\end{itemize}

\subsection{Acronyms}
For acronyms you should use the \emph{glossaries} package and put your acronyms
in the \emph{FrontBackmatter/acronyms.tex} file. The first acronym is always
written in it's long form, the following occurrences are abbreviated: first
occurrence \gls{SNR}, second occurrence \gls{SNR}, plural \glspl{SNR}.

\subsection{Examples on Figures}

\sloppy
When using figures, use vector graphics whenever possible. In
\autoref{fig:wiretapchannel} and \autoref{fig:example} are some examples to
generate vector graphics directly from \LaTeX code. The second example is based
on the \emph{matlab2tikz} script for matlab. You find an example in the
\mbox{\emph{gfx/matlab/create\_example\_graph.m}} file. TikZ is used to generate
the graphics. As it takes some time and memory to recompile a graphic, pdflatex
caches generated figures when the \lstinline|--enable-write18| switch is set
when calling \lstinline|pdflatex|. Graphics are only recompiled when you
uncomment the \lstinline|\tikzset{external/remake next}| command. Figures should
always appear after the first reference in the text or at the top of the same
page as the reference, but never before the reference. Prefer placing figures on
separate pages. Try to always have figures and text on each page. Or place
enough figures to fill a page only with figures.

\begin{figure}
\centering
\begin{tikzpicture}[node distance=6mm]
\node[dspsquare,minimum height=3.2em, minimum width=5em,text height=1em, fill=white]
		(source) {Source};
\node[dspsquare,minimum height=3.2em, minimum width=5em,text height=1em, fill=white, right=of source]
		(encoder) {Encoder};
\node[dspsquare,minimum height=3.2em, minimum width=7em,text height=2em, fill=white, right=of encoder]
		(mainch) {Main Channel\\$Q_M$};
\node[dspnodefull, right=of mainch] (n1) {};
\node[dspsquare,minimum height=3.2em, minimum width=5em,text height=1em, fill=white, right=of n1]
		(decoder) {Decoder};
\node[coordinate,right=of decoder] (n2) {};
\node[dspsquare,minimum height=3.2em, minimum width=10em,text height=2em, fill=white, below=1cm of n1]
		(wiretapch) {Wiretap Channel\\$Q_W$};
\node[coordinate,below=of wiretapch] (n3) {};

\draw[dspconn] (source) -- node[midway,above] {$S^K$} (encoder);
\draw[dspconn] (encoder) -- node[midway,above] {$X^N$} (mainch);
\draw[dspconn] (mainch) -- node[midway,above] {$Y^N$} (decoder);
\draw[dspconn] (decoder) -- node[midway,above] {$S^K$} (n2);
\draw[dspconn] (n1) -- (wiretapch);
\draw[dspconn] (wiretapch) -- (n3) node[below] {$Z^N$};
\end{tikzpicture}
\caption[The wiretap channel]{The wiretap channel (source: \cite{1975:Wyner})}
\label{fig:wiretapchannel}
\end{figure}

\section{Margin Notes}

Especially in the standard SEEMOO template with wide margins, you are
\marginpar{Here you can add text to the margin. For example, to
summarize the section next to it.} encouraged to insert text into the
margins. If you decide to do so, plan to have at least one margin note
per double page.

\section{Some Example Text}
\lipsum[3]

\begin{figure}
\centering
\setlength\figureheight{5cm}
\setlength\figurewidth{0.86\textwidth}
% uncomment the following line to recompile the figure when it changes otherwise a cached version is used
%\tikzset{external/remake next}
% This file was created by matlab2tikz v0.4.7 running on MATLAB 8.1.
% Copyright (c) 2008--2014, Nico Schlmer <nico.schloemer@gmail.com>
% All rights reserved.
% Minimal pgfplots version: 1.3
% 
% The latest updates can be retrieved from
%   http://www.mathworks.com/matlabcentral/fileexchange/22022-matlab2tikz
% where you can also make suggestions and rate matlab2tikz.
% 
\begin{tikzpicture}[%
font=\footnotesize
]

\begin{axis}[%
width=\figurewidth,
height=\figureheight,
scale only axis,
xmin=1,
xmax=100,
xlabel={x axis},
ymin=-1.5,
ymax=1.5,
ylabel={y axis},
legend style={draw=black,fill=white,legend cell align=left},
clip mode=individual,transpose legend,legend columns=2,legend style={at={(0,1)},anchor=north west,draw=black,fill=white,legend cell align=left}
]
\addplot [color=blue,solid]
  table[row sep=crcr]{1	0.309016994374947\\
2	0.587785252292473\\
3	0.809016994374947\\
4	0.951056516295154\\
5	1\\
6	0.951056516295154\\
7	0.809016994374947\\
8	0.587785252292473\\
9	0.309016994374948\\
10	1.22464679914735e-16\\
11	-0.309016994374947\\
12	-0.587785252292473\\
13	-0.809016994374947\\
14	-0.951056516295154\\
15	-1\\
16	-0.951056516295154\\
17	-0.809016994374948\\
18	-0.587785252292473\\
19	-0.309016994374948\\
20	-2.44929359829471e-16\\
21	0.309016994374947\\
22	0.587785252292472\\
23	0.809016994374947\\
24	0.951056516295154\\
25	1\\
26	0.951056516295154\\
27	0.809016994374948\\
28	0.587785252292473\\
29	0.309016994374948\\
30	3.67394039744206e-16\\
31	-0.309016994374947\\
32	-0.587785252292473\\
33	-0.809016994374947\\
34	-0.951056516295153\\
35	-1\\
36	-0.951056516295154\\
37	-0.809016994374948\\
38	-0.587785252292473\\
39	-0.309016994374948\\
40	-4.89858719658941e-16\\
41	0.309016994374945\\
42	0.587785252292473\\
43	0.809016994374947\\
44	0.951056516295153\\
45	1\\
46	0.951056516295154\\
47	0.809016994374947\\
48	0.587785252292474\\
49	0.30901699437495\\
50	6.12323399573677e-16\\
51	-0.309016994374945\\
52	-0.587785252292473\\
53	-0.809016994374948\\
54	-0.951056516295153\\
55	-1\\
56	-0.951056516295154\\
57	-0.809016994374949\\
58	-0.587785252292474\\
59	-0.30901699437495\\
60	-7.34788079488412e-16\\
61	0.309016994374948\\
62	0.587785252292472\\
63	0.809016994374948\\
64	0.951056516295153\\
65	1\\
66	0.951056516295154\\
67	0.809016994374949\\
68	0.587785252292474\\
69	0.309016994374947\\
70	8.57252759403147e-16\\
71	-0.309016994374948\\
72	-0.587785252292472\\
73	-0.809016994374946\\
74	-0.951056516295153\\
75	-1\\
76	-0.951056516295154\\
77	-0.809016994374947\\
78	-0.587785252292474\\
79	-0.309016994374947\\
80	-9.79717439317883e-16\\
81	0.309016994374945\\
82	0.587785252292469\\
83	0.809016994374948\\
84	0.951056516295153\\
85	1\\
86	0.951056516295154\\
87	0.809016994374949\\
88	0.587785252292477\\
89	0.309016994374947\\
90	1.10218211923262e-15\\
91	-0.309016994374945\\
92	-0.587785252292472\\
93	-0.809016994374946\\
94	-0.951056516295154\\
95	-1\\
96	-0.951056516295154\\
97	-0.809016994374949\\
98	-0.587785252292477\\
99	-0.309016994374947\\
100	-1.22464679914735e-15\\
};
\addlegendentry{sinus};

\end{axis}
\end{tikzpicture}%
\caption[Caption for list of figures]{Caption of figure}
\label{fig:example}
\end{figure}

\lipsum[6-10]

%*****************************************
\chapter{Related Work}\label{ch:relatedwork}
%*****************************************
\glsresetall % Resets all acronyms to not used

\lipsum[4]

\cleardoublepage
\ctparttext{The contribution starts with a design chapter, where we mathematically describe the design of the physical layer security system, as well as the adaptive filter of the attacker. After the design follows the implementation on WARP nodes. Here we give an insight into the challenges of implementing the designed MIMO communication system. The last chapter concentrates on evaluating the performance of our proposed attack in simulation and practice.}
\part{Contribution}
%************************************************
\chapter{Design}\label{ch:design}
%************************************************
\glsresetall % Resets all acronyms to not used

\lipsum[9]


%************************************************
\chapter{Implementation}\label{ch:implementation} % $\mathbb{ZNR}$
%************************************************
\glsresetall % Resets all acronyms to not used

\lipsum[4]

%************************************************
\chapter{Evaluation}\label{ch:evaluation} % $\mathbb{ZNR}$
%************************************************
\glsresetall % Resets all acronyms to not used

\lipsum[5]


\cleardoublepage
\ctparttext{After the evaluation, we further discuss the results and give an outlook. In addition, we finish this work with conclusions.}
\part{Discussion and Conclusions}
%************************************************
\chapter{Discussion}\label{ch:Discussion} % $\mathbb{ZNR}$
%************************************************
\glsresetall % Resets all acronyms to not used

\lipsum[6]

%************************************************
\chapter{Conclusions}\label{ch:Conclusions} % $\mathbb{ZNR}$
%************************************************
\glsresetall % Resets all acronyms to not used

\lipsum[7]

% ********************************************************************
% Backmatter
%*******************************************************
\appendix
\cleardoublepage
%\ctparttext{The appendix contains some additional developments, that strongly supported us in the design of our communication system. Furthermore, it contains solutions to some problems.}
\part{Appendix}
%********************************************************************
% Appendix
%*******************************************************
% If problems with the headers: get headings in appendix etc. right
%\markboth{\spacedlowsmallcaps{Appendix}}{\spacedlowsmallcaps{Appendix}}
\chapter{Appendix}\label{ch:Appendix}
\glsresetall % Resets all acronyms to not used

\lipsum[8]
%\include{Chapters/Part04_Chapter04_OOWARPLab}
%********************************************************************
% Other Stuff in the Back
%*******************************************************
\cleardoublepage\include{FrontBackmatter/Bibliography}%\bibliography{Bibliography}
%\cleardoublepage\include{FrontBackmatter/Colophon}
\cleardoublepage% ******************************************************* Declaration
% *******************************************************
\refstepcounter{dummy}
\pdfbookmark[0]{Thesis Statement}{statement} \chapter*{Thesis Statement}
\thispagestyle{empty}
\begingroup
\let\cleardoublepage\relax
\begin{flushright}
	\emph{pursuant to §\,22 paragraph 7 of APB TU Darmstadt}
\end{flushright}
I herewith formally declare that I have written the submitted \myDegree{} independently. I did not use any outside support except for the quoted literature and other sources mentioned in the paper. I clearly marked and separately listed all of the literature and all of the other sources which I employed when producing this academic work, either literally or in content. This thesis has not been handed in or published before in the same or similar form.
In the submitted thesis the written copies and the electronic version are identical in content.

\bigskip

\noindent\textit{\myLocation, \myTime}

\smallskip

\begin{flushright}
	\begin{tabular}{m{5cm}}
		\\ \hline
		\centering\myName \\
	\end{tabular}
\end{flushright}

\vfill

\selectlanguage{ngerman}
\pdfbookmark[0]{Erklärung zur Abschlussarbeit}{erklaerung} \chapter*{Erklärung zur Abschlussarbeit}
\begin{flushright}
	\emph{gemäß §\,22 Abs.\,7 APB der TU Darmstadt}
\end{flushright}
Hiermit versichere ich die vorliegende \myDegree{} ohne Hilfe Dritter und nur mit den angegebenen Quellen und Hilfsmitteln angefertigt zu haben. Alle Stellen, die Quellen entnommen wurden, sind als solche kenntlich gemacht worden. Diese Arbeit hat in gleicher oder ähnlicher Form noch keiner Prüfungsbehörde vorgelegen.
In der abgegebenen Thesis stimmen die schriftliche und elektronische Fassung überein.

\bigskip
 
\noindent\textit{\myLocation, \myTime}

\smallskip

\begin{flushright}
    \begin{tabular}{m{5cm}}
        \\ \hline
        \centering\myName \\
    \end{tabular}
\end{flushright}

\selectlanguage{american}

% ********************************************************************
% Game Over: Restore, Restart, or Quit?
%*******************************************************
\end{document}
% ********************************************************************
