%% -----------------------------------------------------------------------------
% Abstract

%% -----------------------------------------------------------------------------
% English version

\pdfbookmark[1]{Abstract}{Abstract}
\begingroup
\let\clearpage\relax
\let\cleardoublepage\relax
\let\cleardoublepage\relax

\chapter*{Abstract}

With wireless mobile IEEE 802.11 a/g networks, collisions are currently inevitable despite effective counter measures. This work proposes an approach to detect the MAC addresses of transmitting stations in case of a collision and measures its practical feasibility. Recognizing senders using cross-correlation in the time-domain worked surprisingly well in simulations using AWGN and standard Matlab channel models.\\

Real-world experiments using software defined radios also showed promising results despite to some extent decreased accuracy due to channel effects. During the experiments, various modulation and coding schemes (MCS) and scrambler initialization values were compared. Knowing which senders were transmitting leading to a collision could help to develop new improvements to the 802.11 MAC coordination function, or serve as a feature for learning-based algorithms.


%% -----------------------------------------------------------------------------
% German version

\vfill
\selectlanguage{ngerman}
\pdfbookmark[1]{Zusammenfassung}{Zusammenfassung}
\chapter*{Zusammenfassung}

In drahtlosen mobilen Netzwerken nach den IEEE 802.11 a/g Standards sind Kollisionen trotz wirkungsvoller Gegenmaßnahmen nicht vollständig zu vermeiden. Diese Arbeit stellt einen Ansatz zur Erkennung der MAC-Adressen der beteiligten Sender bei einer Kollision vor und untersucht, inwiefern das Verfahren in der Praxis funktionieren könnte. Über Kreuzkorrelation im Zeitbereich funktionierte die Erkennung in Simulationen unter AWGN und verschiedenen Standard-Kanalmodellen erstaunlich gut.\\

Praktischen Experimente unter realen Einflüssen mit Software Defined Radios zeigten ebenfalls vielversprechende Ergebnisse, wenn auch die Genauigkeit der Erkennung durch Kanaleffekte beeinträchtigt wurde. Bei den Experimenten wurden verschiedene Modulation and Coding Schemes (MCS) und Scrambler-Initialisierungen verglichen. Die Kenntnis über die beteiligten Sender bei einer Kollision könnte zur Verbesserung der Koordinierungsfunktion oder als Feature für lernbasierte Verfahren verwendet werden.


\selectlanguage{american}
\endgroup
\vfill
