%% -----------------------------------------------------------------------------

\chapter{Conclusion}\label{ch:Conclusion}
\glsresetall % Resets all acronyms to not used
\glsunset{IEEE} \glsunset{MAC}

Recognizing senders by correlating their \gls{MAC} addresses in an IEEE 802.11a/g frame collision is an approach that to my knowledge has not been explored before. In this thesis, I introduced an algorithm to apply this technique and developed a proof-of-concept implementation using Matlab with the Communications and WLAN System Toolboxes.\\

The technique promiscuously listens on the wireless network interface to cache \gls{MAC} addresses of stations connected to the network. For each of these addresses, several reference signals are pre-modulated and stored, considering different \glspl{MCS} and scrambler initialization values. When a collision is received, samples containing the \gls{MAC} address are cross-correlated in the time domain. The reference signals with the highest correlations are used to determine the senders participating in the collided transmission. The proposed technique was evaluated in simulations and in practical testbed experiments with 64 real, captured \gls{MAC} addresses from the university eduroam network.\\

In simulations, the proposed technique performed very well even for higher \glspl{MCS}. Varying the scrambler initialization caused severe regression, whereas the destination \gls{MAC} address preceding the sender played a minor role. Simulations with standard channel models showed measurable impact, however up to a certain point sender detection retained a reasonable accuracy.\\

Experiments with real hardware, on three \gls{WARP} \glspl{SDR} in particular, showed promising results. Receiving a collision, correlating the frame preambles, and measuring a delay between two collided frames worked very well. Detecting sender \gls{MAC} addresses at \glspl{MCS} 0 and 1 was similar to simulations and provided reasonable sender guesses. Due to channel effects, the accuracy was however reduced. The algorithm produced no usable results for higher \glspl{MCS}.\\

The proposed algorithm is not directly applicable to current versions of the IEEE 802.11 standard using \gls{MIMO}. This is due to possible spatial fractioning of the time-domain samples containing the sender \gls{MAC} address, rendering naive cross-correlation ineffective.\\

Future work could quantize channel effects on real hardware, and adapt the detection technique to modern IEEE 802.11n/ac/ax standards and \gls{MIMO} transmissions.

In summary, sender detection by using cross-correlation in the time domain worked surprisingly well despite some problems.
